%!TEX encoding = UTF-8 Unicode
% $Id: 17-read_macros.tex 17 2014-03-09 13:05:41Z binghe $

\chapter{读取宏~(read-macro)}
\label{chap:read-macros}
\index{read-macros}

在~Lisp 表达式的一生中,有三个最重要的时刻,分别是读取期~(read-time),编译期~(compile-time)
和运行期~(runtime)。运行期由函数左右。\index{times of evaluation}宏给了我们在编译期对程序做转换的机会。
本章讨论读取宏~(read-macro),它们在读取期发挥作用。

\section{宏字符}
\label{sec:macro_characters}

按照~Lisp 的一般哲学,你可以在很大程度上控制~reader。它的行为是由那些可随时
改变的属性和变量控制的。Reader 可以在几个层面上编程。若要改变其行为,最简单的方式就是定义新的
宏字符。

\emph{宏字符} (macro character) 是一种被~Lisp reader 特殊对待的字符。举个例子,小写字母~\texttt{a}
的处理方式和小写字母~\texttt{b} 是一样的,它们都由常规的处理方式处理。
但左括号就有些不同:它告诉~Lisp 开始读取一
个列表。每个这样的字符都有一个与之关联的函数告诉~Lisp reader 当遇到该字符的时候做什么。
你可以改变一个已有的宏字符的关联函数,或者定义你自己的新的宏字符。

内置函数
~\texttt{set-macro-character}\index{set-macro-character@\texttt{set-macro-character}}
提供了一种定义读取宏的方式。它接受一个字符和一个函数,以后
当~\texttt{read}\index{read@\texttt{read}} 遇到这个字符时,它就返回调用该函数的结果。

\begin{figure}
\begin{lstlisting}
(set-macro-character #\'
  #'(lambda (stream char)
      (declare (ignore char))
      (list 'quote (read stream t nil t))))
\end{lstlisting}
  \caption{\texttt{'} 的可能定义}
  \label{fig:possible_definition_of_quote}
\end{figure}

Lisp 中最古老的读取宏之一是~\texttt{'},即引用。你也可以不用~\texttt{'} 而总是将
~\texttt{'a} 写成~\texttt{(quote a)},但这将会非常烦人而且会降低代码的可读性。
引用读取宏使~\texttt{(quote a)} 可以简写成~\texttt{'a}。我们可以用图
~\ref{fig:possible_definition_of_quote} 中的方法实现它。当~\texttt{read} 在一个
普通的上下文中~(例如,不在~\texttt{"a'b"} 或~\texttt{|a'b|} 中) 遇到
~\texttt{'}\index{'@\texttt{'}} 时,它将返回在当前流和字符上调用这个函数的结果。(该函数忽略
了它的第二个形参,因为它总是那个引用字符。) 所以当~\texttt{read} 看到~\texttt{'a} 时,它将返回~\texttt{(quote a)}。

\texttt{read} 的最后三个参数分别控制:是否在碰到~end-of-file\index{end-of-file (eof)} 时报错,
如果不报错的话返回什么值,以及这个~\texttt{read} 调用是否是发生在~\texttt{read} 调用中的
\footnote{译者注:关于~\texttt{read} 的最后一个参
  数~(\texttt{recursive-p}),详见~\textsc{clhs} 中对~\texttt{read} 的解
  释。}。在几乎所有的读取宏里,第二和第四个参数都应该是~\texttt{t},所
以第三个参数也就无关紧要了。

读取宏和常规宏一样,其实质都是函数。
和生成宏展开的函数一样,和宏字符相关的函数,除了作用于它读取的流以外,不应该再有其他副作用。
Common Lisp 明确声明:一个与宏字符相关联的函数何时被执行,或者被执行几次 \pozhehao{} Common Lisp 对其将不给予保证。
(见~\textsc{cltl}2 的~543 页。)

宏和读取宏在不同的阶段分析和观察你的程序。
宏在程序中发生作用时,它已经被~reader 解析成了~Lisp 对象,而读取宏在程序还是文本的阶段时,就对它施加影响了。
尽管如此,通过在这些文本上调用~\verb|read|,一个读取宏,如果它愿意的话,
同样可以得到解析后的~Lisp 对象。这样说来,读取宏至少和常规宏一样强有力。

事实上,读取宏至少在两方面比常规宏更为强大。读取宏可以影响~Lisp 读取的每一样东西,而
宏只是在代码里被展开。并且,由于读取宏通常递归地调用~\verb|read|,一个类似
\begin{lstlisting}
''a
\end{lstlisting}
的表达式将变成
\begin{lstlisting}
(quote (quote a))
\end{lstlisting}
而如果我们试图用一个普通的宏来为~\verb|quote| 定义\abbrev{}的话,
\begin{lstlisting}
(defmacro q (obj)
  `(quote ,obj))
\end{lstlisting}
它在某些情况下可以正常工作,
\begin{lstlisting}
> (eq 'a (q a))
T
\end{lstlisting}
但在被嵌套使用时就不行了\footnote{译者注:解决这个问题的正确方法是定义一个编译器宏
  ~(compiler-macro)。Common Lisp 内置的~\texttt{define-compiler-macro} 用于定义
  编译器宏,详见~\textsc{clhs} 中关于此操作符的说明。}。例如,
\begin{lstlisting}
(q (q a))
\end{lstlisting}
将展开成
\begin{lstlisting}
(quote (q a))
\end{lstlisting}

\section{dispatching 宏字符}
\label{sec:dispatching_macro_characters}

\sq 和其他~\verb|#|\index{sharp(\texttt{\#})} 开头的读取宏一样,是一种称为\emph{dispatching}
读取宏的实例。这些读取宏以两个字符出现,其中第一个字符称为~dispatch 字符。
这类宏的目的,简单说就是尽可能地充分利用~\textsc{ascii} 字符集;如果只有单字符读取宏
的话,那么读取宏的数量就会受限于字符集的大小。

你可以~(通过使用~\verb|make-dispatch-macro-character|\index{make-dispatch-macro-character@\texttt{make-dispatch-macro-character}}) 来定义你自己的~dispatching 宏字符,
但由于~\verb|#| 已经定义了,所以你也可以直接用它。一些~\verb|#|
打头的组合就是特意为你保留的;其他的那些,如果~Common Lisp 还没有给它们赋予含义的话,也可以拿来用。
完整的列表可见~\textsc{cltl}2 的第~531 页。

\begin{figure}
\begin{lstlisting}
(set-dispatch-macro-character #\# #\?
  #'(lambda (stream char1 char2)
      (declare (ignore char1 char2))
        `#'(lambda (&rest ,(gensym))
             ,(read stream t nil t))))
\end{lstlisting}
  \caption{一个用于常数函数的读取宏}
  \label{fig:a_read-macro_for_constant_functions}
\end{figure}

新的~dispatching 宏字符组合可以通过调用~\verb|set-dispatch-macro-character| 函数定义,
除了接受两个字符参数以外和~\verb|set-macro-character| 的用法差不多。一个预留给程序员
的组合是~\verb|#?|\index{\#?@\texttt{\#?}}。图~\ref{fig:a_read-macro_for_constant_functions} 显示了
如何将这个组合定义成一个用于常数函数\index{functions 函数!constant}的读取宏。现在~\verb|#?2| 将被读取为一个函数,其
接受任意数量的参数并且返回~2。例如:
\begin{lstlisting}
> (mapcar #?2 '(a b c))
(2 2 2)
\end{lstlisting}
这个例子里定义的新操作符看起来相当无聊,但在使用了很多函数型参数的程序里,常常会用到常数函数。
事实上,有些方言提供了一个名叫~\verb|always|\index{always@\texttt{always}} 的内置函数,专门用来定义它们。

注意到在这个宏字符的定义中使用宏字符是完全没有问题的:和任何~Lisp 表达式一样,当这个定义
被读取以后这些宏字符就都消失了。在~\verb|#?| 的后面使用宏字符也是可以的。
因为~\verb|#?| 的定义调用了~\verb|read|,所以诸如~\verb|'| 和~\verb|#'|
此类宏字符也可以正常使用:
\begin{lstlisting}
> (eq (funcall #?'a) 'a)
T
> (eq (funcall #?#'oddp) (symbol-function 'oddp))
T
\end{lstlisting}

\section{定界符}
\label{sec:delimiters}

\begin{figure}
\begin{lstlisting}
(set-macro-character #\] (get-macro-character #\)))

(set-dispatch-macro-character #\# #\[
  #'(lambda (stream char1 char2)
      (declare (ignore char1 char2))
      (let ((accum nil)
            (pair (read-delimited-list #\] stream t)))
        (do ((i (ceiling (car pair)) (1+ i)))
            ((> i (floor (cadr pair)))
             (list 'quote (nreverse accum)))
          (push i accum)))))
\end{lstlisting}
  \caption{一个定义定界符的读取宏}
  \label{fig:a_read-macro_defining_delimiters}
\end{figure}

除了简单的宏字符,定义得最多的宏字符要算列表定界符了。另一个为用户
预留的组合字符是~\verb|#[|\index{\#[@\texttt{\#[}}。
图~\ref{fig:a_read-macro_defining_delimiters} 给出的例子,显示了把
这个字符定义成一个更复杂的左括号的方法。它定义形如~\verb|#[x y]| 的
表达式,使得这样的表达式被读取为在~\verb|x| 到~\verb|y| 的闭区间上所有整数的列表:
\begin{lstlisting}
> #[2 7]
(2 3 4 5 6 7)
\end{lstlisting}
这个读取宏里,唯一的新东西是对
~\texttt{read-delimited-list}\index{read-delimited-list@\texttt{read-delimited-list}}
的调用,这个函数是一个完全为这种情况度身定制的内置函数。它的第一个参数是
那个被当作列表结尾的字符。有其名才能行其实,为了把~\verb|]| 识别成定界符,
程序在开始的地方调用了~\verb|set-macro-character|。

\begin{figure}
\begin{lstlisting}
(defmacro defdelim (left right parms &body body)
  `(ddfn ,left ,right #'(lambda ,parms ,@body)))

(let ((rpar (get-macro-character #\))))
  (defun ddfn (left right fn)
    (set-macro-character right rpar)
    (set-dispatch-macro-character #\# left
      #'(lambda (stream char1 char2)
          (declare (ignore char1 char2))
          (apply fn
                 (read-delimited-list right stream t))))))
\end{lstlisting}
  \caption{一个用于定义定界符读取宏的宏}
  \label{fig:a_macro_for_defining_delimiter_read-macros}
  \index{defdelim@\texttt{defdelim}}
\end{figure}

多数潜在的定界符读取宏都将在很大程度上重复图~\ref{fig:a_read-macro_defining_delimiters} 中的代码。
或许可以写个宏,让它从这些机制中提炼出更抽象的接口,以简化代码。
图~\ref{fig:a_macro_for_defining_delimiter_read-macros} 就是一个实现,我们可以像它那样定义
一个\utility{},用其定义定界符读取宏。宏~\verb|defdelim| 接受两个字符,一个参数列表,以及
一个代码主体。参数列表和代码主体隐式地定义了一个函数。一个对~\verb|defdelim| 的调用
将首个字符定义为~dispatching 读取宏,它读取到第二个字符为止,然后将这个函数应用到它读到的
东西,并返回其结果。

无独有偶,图~\ref{fig:a_read-macro_defining_delimiters} 中的函数体也迫切需要一个\utility{},
事实上,这个\utility{}已经定义过了:见~\pageref{fig:mapping_functions} 页的~\verb|mapa-b|。
使用~\verb|defdelim| 和~\verb|mapa-b|,
图~\ref{fig:a_read-macro_defining_delimiters} 中定义的读取宏现在只需写成:
\begin{lstlisting}
(defdelim #\[ #\] (x y)
  (list 'quote (mapa-b #'identity (ceiling x) (floor y))))
\end{lstlisting}

定界符读取宏也可以用来做函数复合\index{functions 函数!composition of}。第~\ref{sec:composing_functions}
节定义了一个用于函数复合的操作符:
\begin{lstlisting}
> (let ((f1 (compose #'list #'1+))
        (f2 #'(lambda (x) (list (1+ x)))))
    (equal (funcall f1 7) (funcall f2 7)))
T
\end{lstlisting}
当我们复合像~\verb|list| 和~\verb|1+| 这样的内置函数时,没有理由等到运行期才去
对~\verb|compose| 的调用求值。第~\ref{sec:when_to_build_functions} 节建议一个
替代方案;通过给一个~\verb|compose| 表达式前缀~sharp-dot 读取宏,
\begin{lstlisting}
#.(compose #'list #'1+)
\end{lstlisting}
我们可以令其在读取期就被求值\index{times of evaluation}。

\begin{figure}
\begin{lisp}
(defdelim #\{ #\} (&rest args)
  `(fn (compose ,@args)))
\end{lisp}
  \caption{一个用于函数型复合的读取宏}
  \label{fig:a_read-macro_for_functional_composition}
  \index{\#@\texttt{\#}}
  % xxx failed to put `{' in index
\end{figure}

这里我们给出一个与之类似但更清晰的解决方案。
图~\ref{fig:a_read-macro_for_functional_composition} 中定义的读取宏定义了一个
~\texttt{\#\{$f_1f_2\ldots f_n$\}} 形式的表达式,这个表达式将被读取成~$f_1,f_2,\ldots,f_n$ 的复合。
这样:
\begin{lisp}
> (funcall #{list 1+} 7)
(8)
\end{lisp}

它生成一个对~\verb|fn|\index{fn@\texttt{fn}} (\pageref{fig:general_function-building_macro} 页)
的调用,该调用在编译期创建函数。

\section{这些发生于何时}
\label{sec:when_what_happens}

最后,澄清一个可能造成困惑的问题应该会有所帮助。如果读取宏是在常规宏之前作用的话,那么
宏是怎样展开成含有读取宏的表达式的呢?例如,这个宏:
\begin{lstlisting}
(defmacro quotable ()
  '(list 'able))
\end{lstlisting}
会生成一个带有引用的展开式。还是说它没有生成?事实上,真相是:这个宏定义中的两个引用在这个
~\texttt{defmacro} 表达式被读取时,就都被展开了,展开结果如下
\begin{lstlisting}
(defmacro quotable ()
  (quote (list (quote able))))
\end{lstlisting}
通常,在宏展开式里包含读取宏是没有什么问题的。因为一个读取宏的定义在读取期和
编译期之间将不会~(或者说不应该) 发生变化。

%%% Local Variables:
%%% coding: utf-8
%%% mode: latex
%%% TeX-master: "onlisp-cn"
%%% End:
