%!TEX encoding = UTF-8 Unicode
% $Id: 14-anaphoric_macros.tex 18 2014-03-12 22:35:24Z binghe $

\chapter{指代宏}
\label{chap:anaphoric_macros}
\index{macros 宏!anaphoric 指代}

第~\ref{chap:variable_capture} 章只是把变量捕捉视为一种问题\pozhehao{}某种
意料之外,并且只会捣乱的负面因素。本章将显示变量捕捉也可以被有建设性地使用。
如果没有这个特性,一些有用的宏就无法写出来。

在~Lisp 程序里,下面这种需求并不鲜见:希望检查一个表达式的返回值是否为非空,如果是的话,使用这个值做某些事。倘若求值表达式的代价比较大,那么通常必须这样做:
\begin{lstlisting}
(let ((result (big-long-calculation)))
  (if result
      (foo result)))
\end{lstlisting}
难道就不能简单一些,让我们像英语里那样,只要说:
\begin{lstlisting}
(if (big-long-calculation)
    (foo it))
\end{lstlisting}
通过利用变量捕捉,我们可以写一个~\texttt{if},让它以这种方式工作。

\section{指代的种种变形}
\label{sec:anaphoric_variants}

在自然语言里,\emph{指代}~(anaphor)\index{anaphor 指代|see{macros!anaphoric}} 是一种引用对话中曾提及事物的表达方式。英语中最常用的
代词可能要算~``it'' 了,就像在~``Get the wrench and put it on the table (拿个
扳手,然后把它放在桌上)'' 里那样。指代给日常语言带来了极大的便利\pozhehao{}试想一下没有它会发生什么\pozhehao{}但它在编程语言里却很少见。这在很大程度上是为了语言着想。指代表达式常会产生歧义,而当今的编程语言从设计上就无法处理这种二义性。

尽管如此,在~Lisp 程序中引入一种形式非常有限的代词,同时避免歧义,还是有可能的。代词,实际上是一种可捕捉的符号。我们可以通过指定某些符号,让它们充当代词,然后再编写宏有意地捕捉\index{capture 捕捉!intentional}这些符号,用这种方式来使用代词。

在新版的~\verb|if| 里,符号~\verb|it| 就是那个我们想要捕捉的对象。Anaphoric
if,简称~\texttt{aif},其定义如下:
\begin{lstlisting}
(defmacro aif (test-form then-form &optional else-form)
  `(let ((it ,test-form))
     (if it ,then-form ,else-form)))
\end{lstlisting}
并如前例中那样使用它:
\begin{lstlisting}
(aif (big-long-calculation)
     (foo it))
\end{lstlisting}
当你使用~\verb|aif| 时,符号~\verb|it| 会被绑定到测试表达式返回的结果。在宏
调用中,\verb|it| 看起来是自由的,但事实上,在~\verb|aif| 展开时,表达式~\verb|(foo it)| 
会被插入到一个上下文中,而~\verb|it| 的绑定就位于该上下文:
\begin{lstlisting}
(let ((it (big-long-calculation)))
  (if it (foo it) nil))
\end{lstlisting}
这样一个在源代码中貌似自由的符号就被宏展开绑定了。本章里所有的指代宏都使用了
这种技术,并加以变化。

\begin{figure}
\begin{lstlisting}
(defmacro aif (test-form then-form &optional else-form)
  `(let ((it ,test-form))
     (if it ,then-form ,else-form)))

(defmacro awhen (test-form &body body)
  `(aif ,test-form
        (progn ,@body)))

(defmacro awhile (expr &body body)
  `(do ((it ,expr ,expr))
       ((not it))
     ,@body))

(defmacro aand (&rest args)
  (cond ((null args) t)
        ((null (cdr args)) (car args))
        (t `(aif ,(car args) (aand ,@(cdr args))))))

(defmacro acond (&rest clauses)
  (if (null clauses)
      nil
      (let ((cl1 (car clauses))
            (sym (gensym)))
        `(let ((,sym ,(car cl1)))
           (if ,sym
               (let ((it ,sym)) ,@(cdr cl1))
               (acond ,@(cdr clauses)))))))
\end{lstlisting}
  \caption{Common Lisp 操作符的指代变形}
  \label{fig:anaphoric_variants_of_common_lisp_operators}
  \index{aif@\texttt{aif}}
  \index{awhen@\texttt{awhen}}
  \index{awhile@\texttt{awhile}}
  \index{aand@\texttt{aand}}
  \index{acond@\texttt{acond}}
\end{figure}

图~\ref{fig:anaphoric_variants_of_common_lisp_operators}\footnote{原书勘误:
  \texttt{(acond (3))} 将返回~\texttt{nil} 而不是~3。
  后面的~\texttt{acond2} 也有同样的问题。} 包含了一些~Common Lisp
操作符的指代变形。\texttt{aif} 下面是~\texttt{awhen},很明显它是~\texttt{when} 的
指代版本:
\begin{lstlisting}
(awhen (big-long-calculation)
  (foo it)
  (bar it))
\end{lstlisting}

\verb|aif| 和~\verb|awhen| 都是经常会用到的,但~\verb|awhile| 可
能是这些指代宏中的唯一一个,被用到的机会比它的正常版的同胞兄
弟~\verb|while| (定义于~\pageref{macro:while} 页) 更多的宏。一般来
说,如果一个程序需要等待~(poll) 某个外部数据源的话,类似~\verb|while|
和~\verb|awhile| 这样的宏就可以派上用场了。而且,如果你在等待一个数据
源,除非你想做的仅是静待它改变状态,否则你肯定会想用从数据源那里
获得的数据做些什么:
\begin{lstlisting}
(awhile (poll *fridge*)
  (eat it))
\end{lstlisting}

% xxxxxxxxxxxxxxxx
\verb|aand| 的定义和前面的几个宏相比之下更复杂一些。它提供了一个~\verb|and| 的指代
版本;每次求值它的实参,\verb|it| 都将被绑定到前一个参数返回的值上。\footnote{
尽管人们喜欢把~\texttt{and} 和~\texttt{or} 相提并论,但实现指代版本的
~\texttt{or} 没有什么意义。一个~\texttt{or} 表达式中的实参只有当它前面的实参求值到
~\texttt{nil} 才会被求值,所以~\texttt{aor} 中的代词将毫无用处。}
在实践中,\texttt{aand} 倾向于在那些做条件查询\index{queries 查询!conditional 条件}的程序中使用,例如这里:
\begin{lstlisting}
(aand (owner x) (address it) (town it))
\end{lstlisting}
它返回~\texttt{x} 的拥有者~(如果有的话) 的地址~(如果有的话) 所属的城镇~(如果有的话)。
如果不使用~\texttt{aand},该表达式就只能写成
\begin{lstlisting}
(let ((own (owner x)))
  (if own
      (let ((adr (address own)))
        (if adr (town adr)))))
\end{lstlisting}

从~\verb|aand| 的定义可以看出,它的展开式将随宏调用中的实参的数量而变。如果没有实参,那么
~\verb|aand|,将像正常的~\verb|and| 那样,应该直接返回~\verb|t|。否则会递归地\index{recursion 递归!in macros}生成
展开式,每一步都会在嵌套的~\verb|aif| 链中产生一层:
\begin{lstlisting}
(aif $\langle\mathrm{first argument}\rangle$
     $\langle\mathrm{expansion for rest of arguments}\rangle$)
\end{lstlisting}
\verb|aand| 的展开必须在只剩下一个实参时终止,而不是像大多数递归函数那样继续展开,直到
~\verb|nil| 才停下来。倘若递归过程一直进行下去,直到消去所有的合取式,那么最终的展开式将总是下面的模样:
\begin{lstlisting}
(aif $\langle{}c_1\rangle$
     $\vdots$
     (aif $\langle{}c_n\rangle$
          t)...)
\end{lstlisting}
这样的表达式会一直返回~\verb|t| 或者~\verb|nil|,因而上面的示例将无法正常工作。

第~\ref{sec:recursion} 节曾警告过:如果一个宏总是产生包含对其自身调用的展开式,那么展开
过程将永不终止。虽然~\verb|aand| 是递归的,但是它却没有这个问题,因为在基本情形里它的展开式没有引用
~\texttt{aand}。

最后一个例子是~\verb|acond|,它用于~\verb|cond| 子句的其余部分想使用测试表达式的返回值
的场合。(这种需求非常普遍,以至于~Scheme 专门提供了一种方式来使用~\verb|cond|\index{Scheme!cond@\texttt{cond}}
子句中测试表达式的返回值。)

在~\verb|acond| 子句的展开式里,测试结果一开始时将被保存在一个由~gensym 生成的变量
里,目的是为了让符号~\verb|it| 的绑定只在子句的其余部分有效。当宏创建这些绑定时,它们应该
总是在尽可能小的作用域里完成这些工作。这里,要是我们省掉了这个~gensym,同时直接把~\verb|it| 绑定到测试表达式
的结果上,就像这样:
\begin{lstlisting}[escapechar=\~]
(defmacro acond (&rest clauses)~\hfill~; wrong
  (if (null clauses)
      nil
      (let ((cl1 (car clauses)))
        `(let ((it ,(car cl1)))
           (if it
               (progn ,@(cdr cl1))
               (acond ,@(cdr clauses)))))))
\end{lstlisting}
那么~\texttt{it} 绑定的作用域也将包括\emph{后续}的测试表达式。

\begin{figure}
\begin{lstlisting}
(defmacro alambda (parms &body body)
  `(labels ((self ,parms ,@body))
     #'self))

(defmacro ablock (tag &rest args)
  `(block ,tag
     ,(funcall (alambda (args)
                 (case (length args)
                   (0 nil)
                   (1 (car args))
                   (t `(let ((it ,(car args)))
                         ,(self (cdr args))))))
               args)))
\end{lstlisting}
  \caption{更多的指代变形}
  \label{fig:more_anaphoric_variants}
  \index{alambda@\texttt{alambda}}
  \index{ablock@\texttt{ablock}}
\end{figure}

图~\ref{fig:more_anaphoric_variants} 有一些更复杂的指代变形。宏
~\texttt{alambda} 是用来字面引用递归函数的\index{functions 函数!literal!recursive}。不过什么时候会需要字面引用递归函数
呢?我们可以通过带~\sq 的~\lexpr 来字面引用一个函数:
\begin{lstlisting}
#'(lambda (x) (* x 2))
\end{lstlisting}
但正如第~\ref{chap:functions} 章里解释的那样,你不能直接用~\lexpr 来表达递归函数。
代替的方法是,你必须借助~\texttt{labels} 定义一个局部函数。下面这个函数~(来自
~\pageref{fun:count-instances} 页)
\begin{lstlisting}
(defun count-instances (obj lsts)
  (labels ((instances-in (lst)
             (if (consp lst)
                 (+ (if (eq (car lst) obj) 1 0)
                    (instances-in (cdr lst)))
                 0)))
    (mapcar #'instances-in lsts)))
\end{lstlisting}
接受一个对象和列表,并返回一个由列表中每个元素里含有的对象个数所组成的数列:
\begin{lstlisting}
> (count-instances 'a '((a b c) (d a r p a) (d a r) (a a)))
(1 2 1 2)
\end{lstlisting}
通过代词,我们可以将这些代码变成字面递归函数。\verb|alambda| 宏使用
~\verb|labels| 来创建函数,例如,这样就可以用它来表达阶乘函数:
\begin{lstlisting}
(alambda (x) (if (= x 0) 1 (* x (self (1- x)))))
\end{lstlisting}
使用~\verb|alambda| 我们可以定义一个等价版本的~\verb|count-instances|,如下:
\begin{lstlisting}
(defun count-instances (obj lists)
  (mapcar (alambda (list)
            (if list
                (+ (if (eq (car list) obj) 1 0)
                   (self (cdr list)))
                0))
          lists))
\end{lstlisting}

\verb|alambda| 与图~\ref{fig:anaphoric_variants_of_common_lisp_operators} 和
~\ref{fig:more_anaphoric_variants} 里的其他宏不一样,后者捕捉的是~\verb|it|,
而~\verb|alambda| 则捕捉~\verb|self|。\verb|alambda| 实例会展开进一个
~\verb|labels| 表达式,在这个表达式中,\verb|self| 被绑定到正在定义的函数上。\verb|alambda|
表达式不但更短小,而且看起来很像我们熟悉的~\verb|lambda| 表达式,这让使用~\verb|alambda| 表达式的代码
更容易阅读。

这个新宏被用了在~\verb|ablock| 的定义里,它是内置的~\verb|block| special form 的一个
指代版本。在~\verb|block| 里面,参数从左到右求值。在~\verb|ablock|
里也是一样,只是在这里,每次求值时变量~\verb|it| 都会被绑定到前一个表达式的值上。

这个宏应谨慎使用。尽管很多时候~\verb|ablock| 用起来很方便,但是它很可
能会把本可以被写得优雅漂亮的函数式程序弄成命令式程序的样子。下面就是一
个很不幸的反面教材:
\begin{lstlisting}
> (ablock north-pole
    (princ "ho ")
    (princ it)
    (princ it)
    (return-from north-pole))
ho ho ho
NIL
\end{lstlisting}

如果一个宏,它有意地使用了变量捕捉,那么无论何时这个宏被导出到另一个包的时候,都必须同时导出那些被捕捉了的符号。
例如,无论~\verb|aif| 被导出到哪里,\verb|it| 也应该同样被导出到同样的地方。否则出现在宏定义里
的~\verb|it| 和宏调用里使用的~\verb|it| 将会是不同的符号。

\section{失败}
\label{sec:failure}
\index{failure}

在~Common Lisp 中符号~\texttt{nil} 身兼三职\index{nil@\texttt{nil}!multiple roles of}。它首先是一个空列表,也就是
\begin{lstlisting}
> (cdr '(a))
NIL
\end{lstlisting}
除了空列表以外,\texttt{nil} 被用来表示逻辑假,例如这里
\begin{lstlisting}
> (= 1 0)
NIL
\end{lstlisting}
最后,函数返回~\texttt{nil} 以示失败。例如,内置~\texttt{find-if}\index{find-if@\texttt{find-if}} 的任务是返回
列表中第一个满足给定测试条件的元素。如果没有发现这样的元素,\texttt{find-if} 将返回
~\texttt{nil}:
\begin{lstlisting}
> (find-if #'oddp '(2 4 6))
NIL
\end{lstlisting}
不幸的是,我们无法分辨出这种情形:即~\texttt{find-if} 成功返回,而成功的原因是它发现了~\texttt{nil}:
\begin{lstlisting}
> (find-if #'null '(2 nil 6))
NIL
\end{lstlisting}

在实践中,用~\texttt{nil} 来同时表示假和空列表并没有招致太多的麻烦。事实上,这样可能相当方便。
然而,用~\texttt{nil} 来表示失败却是一个痛处。因为它意味着一个像
~\texttt{find-if} 这样的函数,其返回的结果可能是有歧义的\index{macros 宏!anaphoric 指代!for distinguishing failure from falsity 为了区别失败和逻辑假}。

对于所有进行查找操作的函数,都会遇到如何区分失败和~\texttt{nil} 返回值的问题。为了解决这个问
题,Common Lisp 至少提供了三种方案。在多重返回值出现之前,最常用的方法是专门
返回一个列表结构。例如,区分~\texttt{assoc}\index{assoc@\texttt{assoc}} 的失败就没有任何麻烦;当执行成功时它返回
成对的问题和答案:
\begin{lstlisting}
> (setq synonyms '((yes . t) (no . nil)))
((YES . T) (NO))
> (assoc 'no synonyms)
(NO)
\end{lstlisting}
按照这个思路,如果担心~\texttt{find-if} 带来的歧义\index{lists!disambiguating return values with},我们可以用~\texttt{member-if}\index{member-if@\texttt{member-if}}
,它不单单返回满足测试的元素,而是返回以该元素开始的整个~cdr:
\begin{lstlisting}
> (member-if #'null '(2 nil 6))
(NIL 6)
\end{lstlisting}

自从多重返回值\index{multiple values 多值!to distinguish failure from falsity 区别失败和逻辑假}诞生之后,这个问题就有了另一个解决方案:用一个值代表数据,而用第二个值指出成功还是失败。内置的~\texttt{gethash}\index{gethash@\texttt{gethash}} 就以这种方式工作。它总是返回两个值,第二个值代表是否找到了什么东西:
\begin{lstlisting}
> (setf edible                      (make-hash-table)
        (gethash 'olive-oil edible) t
        (gethash 'motor-oil edible) nil)
NIL
> (gethash 'motor-oil edible)
NIL
T
\end{lstlisting}
如果你想要检测所有三种可能的情况,可以用类似下面的写法:
\begin{lstlisting}
(defun edible? (x)
  (multiple-value-bind (val found?) (gethash x edible)
    (if found?
        (if val 'yes 'no)
        'maybe)))
\end{lstlisting}
这样就可以把失败和逻辑假区分开了:
\begin{lstlisting}
> (mapcar #'edible? '(motor-oil olive-oil iguana))
(NO YES MAYBE)
\end{lstlisting}

Common Lisp 还支持第三种表示失败的方法:让访问函数接受一个特殊对象作为参数,一般是用个~gensym,
然后在失败的时候返回这个对象\index{gensym@\texttt{gensym}!to indicate failure 表示失败}。这种方法被用于~\texttt{get},它接受一个可选参数来表示当特定属性没有找到时返回的东西:
\begin{lstlisting}
> (get 'life 'meaning (gensym))
#:G618
\end{lstlisting}
\index{life, meaning of}

如果可以用多重返回值,那么~\texttt{gethash} 用的方法是最清楚的。我们不愿意
像调用~\texttt{get} 那样,为每个访问函数都再传入一个参数。并且和另外两种替代方法相比,
使用多重返回值更通用;可以让~\texttt{find-if} 返回两个值,而~\texttt{gethash}
却不可能在不做~consing 的情况下被重写成返回无歧义的列表。这样在编写新的用于查询的函数,
或者对于其他可能失败的任务时,通常采用~\texttt{gethash} 的方式会更好一些。

\begin{figure}
\begin{lstlisting}
(defmacro aif2 (test &optional then else)
  (let ((win (gensym)))
    `(multiple-value-bind (it ,win) ,test
       (if (or it ,win) ,then ,else))))

(defmacro awhen2 (test &body body)
  `(aif2 ,test
         (progn ,@body)))

(defmacro awhile2 (test &body body)
  (let ((flag (gensym)))
    `(let ((,flag t))
       (while ,flag
         (aif2 ,test
               (progn ,@body)
               (setq ,flag nil))))))

(defmacro acond2 (&rest clauses)
  (if (null clauses)
      nil
      (let ((cl1 (car clauses))
            (val (gensym))
            (win (gensym)))
        `(multiple-value-bind (,val ,win) ,(car cl1)
           (if (or ,val ,win)
               (let ((it ,val)) ,@(cdr cl1))
               (acond2 ,@(cdr clauses)))))))
\end{lstlisting}
  \caption{多值指代宏}
  \label{fig:multiple-value_anaphoric_macros}
  \index{aif2@\texttt{aif2}}
  \index{awhen2@\texttt{awhen2}}
  \index{awhile2@\texttt{awhile2}}
  \index{acond2@\texttt{acond2}}
  \index{macros 宏!anaphoric 指代!mutiple-valued 多值}
\end{figure}

在~\texttt{edible?} 里的写法不过相当于一种记帐的操作,它被宏很好地隐藏了起来。对于类似
~\texttt{gethash} 这样的访问函数,我们会需要一个新版本的~\texttt{aif},它绑定和
测试的对象不再是同一个值,而是绑定第一个值,并测试第二个值。这个新版本的~\texttt{aif},称为
~\texttt{aif2},由图~\ref{fig:multiple-value_anaphoric_macros} 给出。使用它,我们
可以将~\texttt{edible?} 写成:
\begin{lstlisting}
(defun edible? (x)
  (aif2 (gethash x edible)
        (if it 'yes 'no)
        'maybe))
\end{lstlisting}

图~\ref{fig:multiple-value_anaphoric_macros} 还包含有~\texttt{awhen},
\texttt{awhile},和~\texttt{acond} 的类似替代版本。作为一个使用~\texttt{acond2}
的例子,见~\pageref{fig:matching_function} 页上~\texttt{match} 的定义。通过使用
这个宏,我们可以用一个~\texttt{cond} 的形式来表达,否则函数将变得更长并且缺少对称性。

\begin{figure}
\begin{lstlisting}
(let ((g (gensym)))
  (defun read2 (&optional (str *standard-input*))
    (let ((val (read str nil g)))
      (unless (equal val g) (values val t)))))

(defmacro do-file (filename &body body)
  (let ((str (gensym)))
    `(with-open-file (,str ,filename)
       (awhile2 (read2 ,str)
         ,@body))))
\end{lstlisting}
  \caption{文件\utility}
  \label{fig:file_utilities}
  \index{do-file@\texttt{do-file}}
\end{figure}

内置的~\texttt{read}\index{read@\texttt{read}} 指示错误的方式和~\texttt{get} 同出一辙。
它接受一个可选参数来说明在遇到~eof 时是否报错,如果不报错的话,将返回何值。图
~\ref{fig:file_utilities} 中给出了另一个版本的~\texttt{read},它用第二个返回值指示失败。
\texttt{read2} 返回两个值,分别是输入表达式和一个标志,如果碰到~eof 的话,这个标志就是~\texttt{nil}。
它把一个~gensym 传给~\texttt{read},万一遇到~eof\index{end-of-file (eof)} 就返回它,这免去了每次调用
~\texttt{read2} 时构造~gensym 的麻烦,这个函数被定义成一个闭包,闭包中带有一个编译期生成的~gensym 的私有拷贝。

图~\ref{fig:file_utilities} 中还有一个宏,它可以方便地遍历一个文件里的所有表达式,
这个宏是用~\texttt{awhile2} 和~\texttt{read2} 写成的。举个例子,借助~\texttt{do-file},我们
可以这样实现~\texttt{load}:
\begin{lstlisting}
(defun our-load (filename)
  (do-file filename (eval it)))
\end{lstlisting}
\index{eval@\texttt{eval}!explicit}

\section{引用透明~(Referential Transparency)}
\label{sec:referential_transparency}
\index{referential transparency 引用透明}

有时认为是指代宏破坏了引用透明\index{macros 宏!anaphoric 指代!and referential transparency 和引用透明},Gelernter\index{Gelernter, David H.}和~Jagannathan\index{Jagannathan, Suresh}是这样定义引用透明的:
\begin{quote}
  一个语言是\emph{引用透明}的,如果~(a) 任意一个子表达式都可以替换成
  另一个子表达式,只要后者和前者的值相等,并且\note{198}~(b) 在给定的
  上下文中,出现不同地方的同一表达式其取值都相同。
\end{quote}

注意到这个标准针对的是语言,而不是程序。没有一个带赋值的语言是引
用透明的\index{assignment 赋值!and referential transparency 和引用透明}。在下面的表达式中
\begin{lstlisting}
(list x
      (setq x (not x))
      x)
\end{lstlisting}
第一个和最后一个~\texttt{x} 带有不同的值,因为被一个~\texttt{setq} 干预了\index{setq@\texttt{setq}!destroys referential transparency}。必须承认,这是丑陋的代码。这一事实意味着~Lisp 不是引用透明的。

Norvig\index{Norvig, Peter} 提到,倘若把~\texttt{if} 重新定义成下面这样将会很方便\note{199}:
\begin{lstlisting}
(defmacro if (test then &optional else)
  `(let ((that ,test))
     (if that ,then ,else)))
\end{lstlisting}
但~Norvig 否定它的理由,也正是因为这个宏破坏了引用透明。

尽管如此,这里的问题在于:上面的宏重定义了内置操作符\index{macros 宏!redefining 重定义!built-in},而不是因为它使用了代词。上面定义中的
~(b) 条款要求一个表达式~``在给定的上下文中'' 必须总是返回相同的值。如果是在这个
~\texttt{let}\index{let@\texttt{let}} 表达式中就没问题了,
\begin{lstlisting}
(let ((that 'which))
  ...)
\end{lstlisting}
符号~\texttt{that} 表示一个新变量,因为~\texttt{let} 就是被用于创建一个新的上下文。

上面那个宏的错误在于,它重定义了~\texttt{if},而~\texttt{if} 的本意\emph{并非}是被用来创建
新的上下文的。如果我们给指代宏取个自己的名字,问题就迎刃而解。(根据~\textsc{cltl}2,
重定义~\texttt{if} 总是非法的。\index{Common Lisp!differences between versions!redefining built-in operators}) 由于~\texttt{aif} 定义的一部分就是建立一个新的上下文,
并且在这个上下文中,\texttt{it} 是一个新变量,所以这样一个宏并没有破坏引用透明。

%% xxx
现在,\texttt{aif} 确实违背了另一个原则,它和引用透明无关:即,不管用什么办法,
新建立的变量都应该在源代码里能很容易地分辨出来。前面的那个
~\texttt{let} 表达式就清楚地表明~\texttt{that} 将指向一个新变量。可能会有反对意见,说:
一个~\texttt{aif} 里面的~\texttt{it} 绑定就没有那么明显。尽管如此,这里有一个不大
站得住脚的理由:\texttt{aif} 只创建了一个变量,并且创建这个变量是我们使用~\texttt{aif} 的唯一
理由。

Common Lisp 自己并没有把这个原则奉为不可违背的金科玉律。\textsc{clos} 函数
~\texttt{call-next-method}\index{call-next-method@\texttt{call-next-method}} 的绑定依赖上下文的方式和~\texttt{aif} 函数体中符号
~\texttt{it} 的绑定方式是一样的。(关于~\texttt{call-next-method} 应如何实现的一个
建议方案,可见~\pageref{fig:defining_methods} 页上的~\texttt{defmeth} 宏。) 在任何
情况下,这类原则的最终目的只有一个:提高程序的可读性。并且代词确实让程序更容易阅读,
正如它们让英语更容易阅读那样。

%%% Local Variables:
%%% coding: utf-8
%%% mode: latex
%%% TeX-master: "onlisp-cn"
%%% End:
