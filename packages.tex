%!TEX encoding = UTF-8 Unicode
% $Id: packages.tex 5 2009-08-07 08:40:43Z binghe $

\chapter{附录: 包~(packages)}
\label{chap:packages}

包~(packages)\index{packages},是~Common Lisp 把代码组织成模块的方式。
早期的~Lisp 方言有一张符号表,即~\emph{oblist}\footnote{译者注:GNU
  Emacs 和~XEmacs 使用的是一张名为~\emph{obarray} 的哈希表。}。在这张
表里列出了系统中所有已经读取到的符号。借助~oblist 里的符号表项,系统得
以存取数据,诸如对象的值,以及属性列表等。保存在~oblist 里的符号被称
为~\emph{interned}\index{interning}。

晚近的~Lisp 方言把~oblist 的概念放到了一个个\emph{包}里面。现在,符号
不仅仅是被~intern 了,而是被~intern 在某个包里。包之所以支持模块化
\index{modularity 模块化}是因为在一个包里的~intern 的符号只有在其被显式
声明为能被其它包访问的时候,它才能为外部访问~(除非用一些歪门邪道的招数)。

包是一种~Lisp 对象。当前包\index{packages!current}常常被保存在一个名
为~\texttt{*package*}\index{package@\texttt{*package*}} 的全局变量里面。当~Common Lisp
启动时,当前包就是用户包\index{packages!user}:或者叫~\texttt{user}\index{user@\texttt{user}}
(\textsc{cltl}1 实现),或者叫~\texttt{common-lisp-user}\index{common-lisp-user@\texttt{common-lisp-user}}
(\textsc{cltl}2 实现)\index{Common Lisp!differences between versions!name of user package}。

包一般用自己的名字相互区别,而这些名字采用的是字符串的形式。要知道当前包的包名,
可以试试:
\begin{verbatim}
> (package-name *package*)
"COMMON-LISP-USER"
\end{verbatim}

通常,当读入一个符号时,它就被~intern 到当前的包里了。要弄清给定
符号所~intern 的是哪个包,我们可以用~\texttt{symbol-package}\index{symbol-package@\texttt{symbol-package}}:
\begin{verbatim}
> (symbol-package 'foo)
#<Package "COMMON-LISP-USER" 4CD15E>
\end{verbatim}
这个返回值是实际的包对象。为便于将来使用,我们给~\texttt{foo} 赋一个值:
\begin{verbatim}
> (setq foo 99)
99
\end{verbatim}

使用~\texttt{in-package}\index{in-package@\texttt{in-package}},我们就可以切换\index{packages!switching}到另一个新的包,若有需要的话这
个包会被创建出来\index{packages!creating}\footnote{在较早期的~Common Lisp 实现下,请省略
  掉~\texttt{:use} 参数}:
\begin{verbatim}
> (in-package 'mine :use 'common-lisp)
#<Package "MINE" 63390E>
\end{verbatim}
此时此刻应该会响起诡异的背景音乐,因为我们已经身处另一个世界:在这
里~\texttt{foo} 已经不似从前了:
\begin{verbatim}
MINE> foo
>>Error: FOO has no global value.
\end{verbatim}
为什么会这样?因为之前被我们设置成~\texttt{99} 的那个~\texttt{foo} 和现
在~\texttt{mine} 里面的这个~\texttt{foo} 是两码事\index{packages!using distinct}。\footnote{有的~Common
  Lisp 实现会在~toplevel 提示符的前面显示包的名字。这个特性不是必须
  的,但的确是比较贴心的设计。}要从用户包之外引用原来的这个~\texttt{foo},
我们必须把包名和两个冒号\index{::@\texttt{::}}作为它的前缀:
\begin{verbatim}
MINE> common-lisp-user::foo
99
\end{verbatim}

因此,具有相同打印名称\index{print-names}\index{symbols 符号!names of}的不同符号得以在不同包中共存。这样就可以在名
为~\texttt{common-lisp-user} 的包里有一个~\texttt{foo},同时
在~\texttt{mine} 包里也有一个~\texttt{foo},并且它们两个是不一样的符号。
实际上,这就是~package 的一部分用意所在,即:你在为你的函数和变量
取名字的同时,就不用担心别人会把一样的名字用在其它东西上。现在,就算有
重名的情况,重名的符号之间也是互不相干的。

与此同时,包也提供了一种信息隐藏的手段\index{hiding implementation details}。对程序来说,它必须使用名字来引
用不同的函数和变量。如果你不让一个名字在你的包之外可见的话,那么另一个
包中的代码就无法使用或者修改这个名字所引用的对象。

在写程序的时候,把包的名字带上两个冒号做为前缀并不是个好习惯。你要是
这样做的话,就违背了模块化\index{modularity 模块化}设计的初衷,而这正是包机制的本意。如果你
不得不使用双冒号来引用一个符号,这应该就是有人根本就不希望你引用它。

一般来说,你只应该引用那些被~\emph{export} 了的符号\index{symbols 符号!exported}。把符号
从它所属的包~export 出来,我们就能让这个符号对其它包变得可见。
要导出一个符号,我们可以调用~(你肯定已经猜到了) \texttt{export}\index{export@\texttt{export}}:
\begin{verbatim}
MINE> (in-package 'common-lisp-user)
#<Package "COMMON-LISP-USER" 4CD15E>
> (export 'bar)
T
> (setq bar 5)
5
\end{verbatim}

现在,如果回到了~\verb|mine| 包,那么就可以用一个冒号引用~\verb|bar|\index{:@\texttt{:}},
因为这个名字是外部可见的:
\begin{verbatim}
> (in-package ’mine)
#<Package "MINE" 63390E>
MINE> common-lisp-user:bar
5
\end{verbatim}

如果把~\texttt{bar} \emph{import}\index{symbols 符号!imported} 到~\texttt{mine} 里面,我们就能更进一步,
让~\texttt{mine} 能和~user 包共享~\texttt{bar} 这个符号:
\begin{verbatim}
MINE> (import 'common-lisp-user:bar)
T
MINE> bar
5
\end{verbatim}
\index{import@\texttt{import}}
在导入~\texttt{bar} 之后,我们可以根本不用加任何包的限定符,就能引用
它了。现在,这两个包共享了同一个符号\pozhehao{}再没有一个独立
的~\texttt{mine:bar} 了。

万一已经有了一个会怎么样呢?在这种情况下,\texttt{import} 调用会导致一
个错误,就像下面我们试着~import \texttt{foo} 时造成的错误一样:
\begin{verbatim}
MINE> (import 'common-lisp-user::foo)
>>Error: FOO is already present in MINE.
\end{verbatim}

之前,我们在~\texttt{mine} 里对~\texttt{foo} 进行了一次不成功的求值,这
次求值顺带着使得一个名为~\texttt{foo} 的符号被加入了~\texttt{mine}。由
于这个符号在全局范围内还没有值,因此产生了一个错误,但是输入符号名字的直接
后果就是使它被~intern 进了这个包。所以,当我们现在想把~\texttt{foo} 引
进~\texttt{mine} 的时候,\texttt{mine} 里面已经有一个相同名字的符号了。

通过让一个包\emph{使用}~(use) 另一个包,我们也能批量的引入符号:
\begin{verbatim}
MINE> (use-package 'common-lisp-user)
T
\end{verbatim}
这样,所有~user package 引出的符号就会自动地被引进到~\texttt{mine} 里面
去了\index{packages!inheriting symbols from}。(要是~user package 已经引出了~\texttt{foo} 的话,这个函数调用也会
出一个错。)

根据~\textsc{cltl}2,包含内建操作符和变量名字的包被称
为~\texttt{common-lisp}\index{common-lisp@\texttt{common-lisp}} 而不是~\texttt{lisp},因此新一些的包在缺省情况
下已不再使用~\texttt{lisp} 包\index{Common Lisp!differences between versions!Lisp package}了。由于我们通过调用~\texttt{in-package}
创建了~\texttt{mine},而在这次调用中也~\texttt{use} 了这个包,所以所
有~Common Lisp 的名字在~\texttt{mine} 中都是可见的:
\begin{verbatim}
MINE> #'cons
#<Compiled-Function CONS 462A3E>
\end{verbatim}
在实际的编程中,你不得不让所有新编写的包使用~\texttt{common-lisp} (或者
其他某个含~Lisp 操作符的包)。否则你甚至会没办法跳出这个新的
包。\index{packages!aberrations involving}\footnote{译者注:即你不仅没有办法使用~\texttt{cons},更糟糕的
  是,你也不能用~\texttt{in-package} 切换到其它包。}

一般来说,在编译后的代码中,不会像刚才这样在顶层进行包的操作。更多的时
候,这些关于包的函数调用会被包含在源文件中。通常,只要
把~\texttt{in-package} 和~\texttt{defpackage}\index{defpackage@\texttt{defpackage}} 放在源文件的开头就可以
了。(\texttt{defpackage}\index{Common Lisp!differences between versions!defpackage@\texttt{defpackage}} 宏是~\textsc{cltl}2 里新引进的,但是有些较老的
实现也提供了它。) 如果你要编写一个独立的包,下面列出了你可能会放在
对应的源文件最开始地方的代码:
\begin{verbatim}
(in-package 'my-application :use 'common-lisp)
(defpackage my-application
            (:use common-lisp my-utilities)
            (:nicknames app)
            (:export win lose draw))
\end{verbatim}
\index{use@\texttt{:use}}
\index{export@\texttt{:export}}
\index{nicknames@\texttt{:nicknames}}
这会使得该文件里所有的代码,或者更准确地说,文件里所有的名字,都纳入
了~\texttt{my-application} 这个包。\texttt{my-application} 同时使用
了~\texttt{common-lisp} 和~\texttt{my-utilities},因此,不用加任何包名
作为前缀,所有被引出的符号都可以直接使用。

\texttt{my-application} 本身仅仅引出了三个符号,它们分别
是:\texttt{win}、\texttt{lose} 和~\texttt{draw}。由于在调
用~\texttt{in-package} 的时候,我们给~\texttt{my-application} 取了一个
绰号\index{packages!nicknames for 绰号}~\texttt{app},在其它包里面的代码可以用类似~\texttt{app:win} 的名字
来引用这些符号。

像这样的用包来提供的模块化的确有点不自然。我们的包里面不是对象,而是一
堆名字。每个使用~\texttt{common-lisp} 的包都引入了~\texttt{cons} 这个名
字,原因在于~\texttt{common-lisp} 包含了一个叫这个名字的函数。但是,这
样会导致一个名字叫~\texttt{cons} 的变量也在每个使
用~\texttt{common-lisp} 的程序里可见。这样的事情同样也会在~Common Lisp 的其他名
字空间\index{name-spaces}重演。如果包~(package) 这个机制让你头痛,那么这就是一个最主要的原因\pozhehao{}包不是基于对象而是基于名字。

和包相关的操作会发生在读取时~(read-time),而非运行时。这可能会造成一些
困扰。我们输入的第二个表达式:
\begin{verbatim}
(symbol-package 'foo)
\end{verbatim}
之所以会返回它返回的那个值是因为:\emph{读取这个查询语句的同时,答案就被生成了}。
为了求值这个表达式,Lisp 必须先读入它,这意味着要~intern~\texttt{foo}。

再来个例子,看看下面把两个表达式交换顺序的结果,这两个表达式前面曾出现过:
\begin{verbatim}
MINE> (in-package 'common-lisp-user)
#<Package "COMMON-LISP-USER" 4CD15E>
> (export 'bar)
\end{verbatim}
通常来说,在顶层输入两个表达式的效果等价于把这两个表达式放在一
个~\texttt{progn} 里面。不过这次有些不同。如果我们这样说
\begin{verbatim}
MINE> (progn (in-package 'common-lisp-user)
             (export 'bar))
>>Error: MINE::BAR is not accessible in COMMON-LISP-USER.
\end{verbatim}
则会得到个错误提示。错误的原因在于~\texttt{progn} 表达式在求值之前
就已经被~\texttt{read} 处理过了。当调用~\texttt{read} 时,当前包还
是~\texttt{mine},因而~\texttt{bar} 被认为是~\texttt{mine:bar}。 运行这
个表达式的效果就好像我们想要从~user 包~export 出~\texttt{mine:bar},而不是
从~\verb|common-lisp-user|~export 出~\texttt{common-lisp-user:bar} 一样。

package 被如此定义,使得编写那些把符号当作数据\index{symbols 符号!as data}的程序成为一桩麻烦事。举个例子,要是像下面那样定义~\texttt{noise}:
\begin{verbatim}
(in-package 'other :use 'common-lisp)
(defpackage other
            (:use common-lisp)
            (:export noise))

(defun noise (animal)
  (case animal
     (dog 'woof)
     (cat 'meow)
     (pig 'oink)))
\end{verbatim}
这样的话,如果我们从另外一个包调用~\texttt{noise},同时传进去的参数是不
认识的符号,\texttt{noise} 会走到~\texttt{case} 语句的末尾,并返
回~\texttt{nil}:
\begin{verbatim}
OTHER> (in-package 'common-lisp-user)
#<Package "COMMON-LISP-USER" 4CD15E>
> (other:noise 'pig)
NIL
\end{verbatim}
这是因为传进去的参数是~\texttt{common-lisp-user:pig} (这没有冒犯
阁下的意思),然而~\texttt{case} 接受~key 是~\texttt{other:pig}。为了
让~\texttt{noise} 像我们期望的那样工作,就必须把里面用到的所有六个符号都
引出来,再在调用~\texttt{noise} 的包里面引入它们。

在此例中,我们也可以通过使用关键字\index{keywords 关键字}而不是常规的符号,来绕过这个问
题。倘若~\texttt{noise} 像下面这样定义
\begin{verbatim}
(defun noise (animal)
 (case animal
   (:dog :woof)
   (:cat :meow)
   (:pig :oink)))
\end{verbatim}
的话,我们就能从任意一个包安全地调用这个函数了:
\begin{verbatim}
OTHER> (in-package 'common-lisp-user)
#<Package "COMMON-LISP-USER" 4CD15E>
> (other:noise :pig)
:OINK
\end{verbatim}

关键字就像金子\index{gold 金子}:普适而且自身就能表明其价值。不论在哪里它们都是可见的,
而且它们从不需要被引用。
在编写类似~\texttt{defanaph} (~\pageref{macro:defanaph} 页) 的符号驱动的函数
时,基本上应该总是用关键字参数。

包里面有很多地方让人不解。这里对这一主题的介绍不过是冰山一角。要知
道所有的细节,请参考~\textsc{cltl}2 的第~11 章。

%%% Local Variables:
%%% coding: utf-8
%%% mode: latex
%%% TeX-master: "onlisp-cn"
%%% End:
