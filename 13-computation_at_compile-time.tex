%!TEX encoding = UTF-8 Unicode
% $Id: 13-computation_at_compile-time.tex 17 2014-03-09 13:05:41Z binghe $

\chapter{编译期计算}
\label{chap:computation_at_compile-time}

前面的章节描述了几类只能用宏实现的操作符。本章将介绍用函数可以解决,但用宏\index{macros 宏!for computation at compile-time 编译器计算}能更高效的一类问题。第~\ref{sec:macro_or_function} 节列出了在给定情形下使用宏的利弊。
有利的一面包括~``在编译期完成计算''。有时,如果把操作符实现成宏,就可以在展开时完成部分工作。
本章会着眼于那些充分利用这种可能性的宏。

\section{新的\utility{}}
\label{sec:new_utility_13}

\begin{figure}
\begin{lstlisting}
(defun avg (&rest args)
  (/ (apply #'+ args) (length args)))

(defmacro avg (&rest args)
  `(/ (+ ,@args) ,(length args)))
\end{lstlisting}
\index{avg@\texttt{avg}}
  \caption{求平均值时转移计算}
  \label{fig:shifting_computation_when_finding_averages}
\end{figure}

第~\ref{sec:macro_or_function} 节里提出,通过宏就可能把计算转移到编译期完成。在那里,我们
曾经把宏~\verb|avg| 作为例子,它会返回其参数的平均值:
\begin{lstlisting}
> (avg pi 4 5)
4.047...
\end{lstlisting}
在图~\ref{fig:shifting_computation_when_finding_averages} 中先把~\verb|avg|
定义成函数,然后又用宏实现它。当把~\verb|avg| 定义成宏时,对~\verb|length| 的调用
可以在编译期完成。在宏版本里我们也避免了在运行期处理~\verb|&rest| 参数的开销。
所以在许多实现里,写成宏的~\verb|avg| 会更快。

\begin{figure}
\begin{lstlisting}
(defun most-of (&rest args)
  (let ((all 0)
        (hits 0))
    (dolist (a args)
      (incf all)
      (if a (incf hits)))
    (> hits (/ all 2))))

(defmacro most-of (&rest args)
  (let ((need (floor (/ (length args) 2)))
        (hits (gensym)))
    `(let ((,hits 0))
       (or ,@(mapcar #'(lambda (a)
                         `(and ,a (> (incf ,hits) ,need)))
                     args)))))
\end{lstlisting}
\index{most-of@\texttt{most-of}}
  \caption{转移和避开计算}
  \label{fig:shifting_and_avoiding_computation}
\end{figure}

这种优化省去了不必要的计算,它的实现归功于在展开期就知道参数的数量。它还可以和我们在~\verb|in|
(\pageref{fig:macros_for_conditional_evaluation} 页) 中进行的另一类优化
结合起来,后者甚至可以避免求值一些参数。
图~\ref{fig:shifting_and_avoiding_computation} 中有两个版本
的~\verb|most-of|,它在多数参数都为真的时候返回真:
\begin{lstlisting}
> (most-of t t t nil)
T
\end{lstlisting}
和~\verb|in| 一样,宏版本展开成的代码只求值它需要数量的参数。例如,
\verb|(most-of (a) (b) (c))| 展开的等价代码:
\begin{lstlisting}
(let ((count 0))
  (or (and (a) (> (incf count) 1))
      (and (b) (> (incf count) 1))
      (and (c) (> (incf count) 1))))
\end{lstlisting}
在最理想的情况下,只对刚过半的参数求值。

\begin{figure}
\begin{lstlisting}
(defun nthmost (n lst)
  (nth n (sort (copy-list lst) #'>)))

(defmacro nthmost (n lst)
  (if (and (integerp n) (< n 20))
      (with-gensyms (glst gi)
        (let ((syms (map0-n #'(lambda (x) (gensym)) n)))
          `(let ((,glst ,lst))
             (unless (< (length ,glst) ,(1+ n))
               ,@(gen-start glst syms)
               (dolist (,gi ,glst)
                 ,(nthmost-gen gi syms t))
               ,(car (last syms))))))
      `(nth ,n (sort (copy-list ,lst) #'>))))

(defun gen-start (glst syms)
  (reverse
    (maplist #'(lambda (syms)
                 (let ((var (gensym)))
                   `(let ((,var (pop ,glst)))
                      ,(nthmost-gen var (reverse syms)))))
             (reverse syms))))

(defun nthmost-gen (var vars &optional long?)
  (if (null vars)
      nil
      (let ((else (nthmost-gen var (cdr vars) long?)))
        (if (and (not long?) (null else))
            `(setq ,(car vars) ,var)
            `(if (> ,var ,(car vars))
                 (setq ,@(mapcan #'list
                                 (reverse vars)
                                 (cdr (reverse vars)))
                       ,(car vars) ,var)
                 ,else)))))
\end{lstlisting}
\index{nthmost@\texttt{nthmost}}
  \caption{使用编译期知道的参数}
  \label{fig:use_of_arguments_known_at_compile-time}
\end{figure}

如果仅仅知道宏的部分参数值,也一样有可能把计算工作转移到编译期进行。图
~\ref{fig:use_of_arguments_known_at_compile-time} 就给出了这样的一个宏。
函数~\verb|nthmost| 接受一个数~$n$ 以及一个数列,并返回数列中第~$n$ 大的数;
和其他序列函数一样,它是从零开始索引的:
\begin{lstlisting}
> (nthmost 2 '(2 6 1 5 3 4))
4
\end{lstlisting}
函数版本写得非常简单。它对列表排序,然后对排序的结果调用~\verb|nth|。由于~\verb|sort| 是
破坏性的,\verb|nthmost| 在排序之前先复制列表。这样实现,使得~\verb|nthmost| 在两方面
影响效率:它构造新的点对,而且对整个参数列表排序,尽管我们只关心前~$n$ 个。

如果我们在编译期知道~$n$ 的值,就可以从另一个角度分析这个问题了。
图~\ref{fig:use_of_arguments_known_at_compile-time} 中的其余代码定义了一个宏版本的
~\verb|nthmost|。这个宏做的第一件事是去检查它的第一个参数。如果第一个参数字面上不是
一个数,它就被展开成和我们上面看到的相同的代码。如果第一个参数是一个数的话,我们可以采用
另一个办法。比方说,如果你想要找到一个盘子里第三大的那块饼干\index{cookies 饼干},那么你可以依次查看每一块
饼干同时保持手里总是拿着已知最大的三块,用这个办法达到目的。当你检查完所有的饼干之后,你手里最小
的那块饼干就是你要找的了。如果~$n$ 是一个小常数,并且这个数字远小于饼干的总数,那么和``先对它们的全部进行排序''
的方法相比,用这种技术可以让你更方便地得到想找的那块饼干。\index{sorting!partial}

\begin{figure}
\begin{lstlisting}
(nthmost 2 nums)
\end{lstlisting}
展开成:
\begin{lstlisting}
(let ((#:g7 nums))
  (unless (< (length #:g7) 3)
    (let ((#:g6 (pop #:g7)))
      (setq #:g1 #:g6))
    (let ((#:g5 (pop #:g7)))
      (if (> #:g5 #:g1)
          (setq #:g2 #:g1 #:g1 #:g5)
          (setq #:g2 #:g5)))
    (let ((#:g4 (pop #:g7)))
      (if (> #:g4 #:g1)
          (setq #:g3 #:g2 #:g2 #:g1 #:g1 #:g4)
          (if (> #:g4 #:g2)
              (setq #:g3 #:g2 #:g2 #:g4)
              (setq #:g3 #:g4))))
    (dolist (#:g8 #:g7)
      (if (> #:g8 #:g1)
          (setq #:g3 #:g2 #:g2 #:g1 #:g1 #:g8)
          (if (> #:g8 #:g2)
              (setq #:g3 #:g2 #:g2 #:g8)
              (if (> #:g8 #:g3)
                  (setq #:g3 #:g8)
                  nil))))
    #:g3))
\end{lstlisting}
  \caption{\texttt{nthmost} 的展开式}
  \label{fig:expansion_of_nthmost}
\end{figure}

这是一种当~$n$ 在编译期已知时采取的策略。在它的展开式里,宏创建了~$n$ 个变量,然后调用
~\verb|nthmost-gen| 来生成那些求值成查看每一块饼干的代码。图
~\ref{fig:expansion_of_nthmost} 给出了一个示例的宏展开。除了不能作为~\verb|apply| 的参数传递以外,
宏~\verb|nthmost| 在行为上和原来的函数一样。这里使用宏的理由完全
是出于效率:宏版本不在运行期构造新点对,并且如果~$n$ 是一个小的常数,那么比较的次数可以更少。

难道为了写出高效的程序,就必须兴师动众,编这么长的宏么?对于本例来说,答案可能是否定的。这里之所以给出两个版本
的~\verb|nthmost|,主要的原因是想举个例子,它揭示了一个普遍的原则:当某些参数在编译期已知时,你可以用宏
来生成更高效的代码。是否利用这种可能性取决于你想获得多少好处,以及你可以另外付出多少努力
来编写一个高效的宏版本。由于~\verb|nthmost| 的宏版本既长又繁,它可能只有在极端场合
才值得去写。尽管如此,就算你宁愿不利用它,编译期已知的信息总还是一个值得考虑的因素。

\section{举例:贝塞尔曲线}
\label{sec:example_bezier_curves}

就像~\verb|with-| 宏~(第~\ref{sec:the_with_macro} 节),用于编译期计算的宏更
像是为特定应用而写的,而不是为通用目的设计的\utility{}。通用的\utility{}在编译期能了解多少信息?
它的参数个数,可能还有某些参数的值。但如果我们想要利用其他的约束条件,这些条件也许就只能是程序自己
才懂得使用的信息了。

本节将作为一个实例,展示宏是如何加速贝塞尔曲线\index{Bezier curves 贝塞尔曲线}的生成的。如果对曲线的操作是交互式的话,那么它们的生成速度必须得非常快才行。
可以看出,如果曲线的分段数是已知的,那么大多数计算就可以在编译期完成。把
曲线生成器写成一个宏,我们就可以将预先算好的值嵌入到代码中。这应该比把它们保存在数组
里,这种更常规的优化方式甚至还要快。

一条贝塞尔曲线由四个点确定\pozhehao{}两个端点和两个控制点。在两维平面上,这些点定义了
曲线上所有点的~$x$ 和~$y$ 坐标的参数方程。如果两个端点是~$(x_0,y_0)$ 和~$(x_3,y_3)$,
以及两个控制点~$(x_1,y_1)$ 和~$(x_2,y_2)$,那么曲线上点的方程就是:
\begin{align*}
  x &= (x_3-3x_2+3x_1-x_0) u^3 + (3x_2-6x_1+3x_0) u^2 + (3x_1-3x_0) u + x_0\\
  y &= (y_3-3y_2+3y_1-y_0) u^3 + (3y_2-6y_1+3y_0) u^2 + (3y_1-3y_0) u + y_0
\end{align*}
如果我们用~$u$ 在~0 和~1 之间的~$n$ 个值来求值这个方程,我们就得到曲线上的~$n$ 个点。
举个例子,如果我们想把曲线画成~20 条线段,那么我们将用~$u=.05,.1,\ldots,.95$ 来求值
方程。对于~$u$ 在~0 或~1 上的求值是不需要的,因为如果~$u=0$ 它们将生成第一个端点
~$(x_0,y_0)$,而当~$u=1$ 时它们将生成第二个端点~$(x_3,y_3)$。

有个显而易见的优化方法是令~$n$ 为定值,并提前计算~$n$ 的指数,然后将它们存在一个~$(n-1)\times 3$
的数组里。通过把曲线生成器定义成一个宏,我们甚至可以做得更好。如果~$n$ 在展开时已知,程序
可以简单地展开成~$n$ 条画线指令。那些预先计算好的~$n$ 的指数,可以直接作为字面上的值插入
到宏展开式里,而不必再保存在数组里了。

\begin{figure}
\begin{lstlisting}
(defconstant *segs* 20)
(defconstant *du* (/ 1.0 *segs*))
(defconstant *pts* (make-array (list (1+ *segs*) 2)))

(defmacro genbez (x0 y0 x1 y1 x2 y2 x3 y3)
  (with-gensyms (gx0 gx1 gy0 gy1 gx3 gy3)
    `(let ((,gx0 ,x0) (,gy0 ,y0)
           (,gx1 ,x1) (,gy1 ,y1)
           (,gx3 ,x3) (,gy3 ,y3))
       (let ((cx (* (- ,gx1 ,gx0) 3))
             (cy (* (- ,gy1 ,gy0) 3))
             (px (* (- ,x2 ,gx1) 3))
             (py (* (- ,y2 ,gy1) 3)))
         (let ((bx (- px cx))
               (by (- py cy))
               (ax (- ,gx3 px ,gx0))
               (ay (- ,gy3 py ,gy0)))
           (setf (aref *pts* 0 0) ,gx0
                 (aref *pts* 0 1) ,gy0)
           ,@(map1-n #'(lambda (n)
                         (let* ((u (* n *du*))
                                (u2 (* u u))
                                (u3 (expt u 3)))
                           `(setf (aref *pts* ,n 0)
                                  (+ (* ax ,u3)
                                     (* bx ,u2)
                                     (* cx ,u)
                                     ,gx0)
                                  (aref *pts* ,n 1)
                                  (+ (* ay ,u3)
                                     (* by ,u2)
                                     (* cy ,u)
                                     ,gy0))))
                     (1- *segs*))
           (setf (aref *pts* *segs* 0) ,gx3
                 (aref *pts* *segs* 1) ,gy3))))))
\end{lstlisting}
  \caption{生成贝塞尔曲线的宏}
  \label{fig:macro_for_generating_bezier_curves}
\end{figure}

图~\ref{fig:macro_for_generating_bezier_curves} 中有一个实现了这一策略的曲线生成
宏。它抛弃了立即画线的策略,而是将生成的点保存在数组里。当交互式地移动一条曲线时,每一个实例
必须画两次:一次显示它,还有一次是在画下一个实例之前清除它。在两次画线之间,这些点必须保存在某个地方。

当~$n=20$ 时,\verb|genbez| 展开成~21 个~\verb|setf|。由于~$u$ 的指数直接出现在
代码里,我们省下了在运行期查找它们的开销,以及在启动时计算它们的开销。和~$u$ 的指数一样,
数组的索引以常量的形式出现在展开式中,所以对那些~\verb|(setf aref)| 的边界检查
\index{compilation 编译!bounds-checking during 期的边界检查}也可以在编译期完成。

\section{应用}
\label{sec:compile-time:applications}

后面的章节将会提到其它一些宏,它们也使用了编译期的已知信息。其中一个很好的例子是~\verb|if-match|
(\pageref{macro:if-match} 页)。在这个例子里,模式匹配器\index{pattern-matching}会比较两个序列,序列中可能含有变量,在比较的过程中,模式匹配器会分析是否
存在某种给这些变量赋值的方式,可以让两个序列相等。\verb|if-match| 的设计显示:如果序列
中的一个在编译期已知,并且只有这个序列里含有变量,那么匹配可以做得更高效。一个办法是在运行期
比较两个序列,并构造列表来保存这个过程中建立的变量绑定,不过我们可以改成用宏,让宏生成的代码
严格按照已知的序列来一一对照比较,同时可以在真正的~Lisp 变量里保存绑定。

第~\ref{chap:a_query_compiler}--\ref{chap:prolog} 章里描述的嵌入式语言\index{embedded languages},也在很大程度
上利用了这些可在编译期获得的信息。由于嵌入式语言就是编译器,利用这些信息是其唯一自然的工作
方式。这是一个普遍规律:越是精心设计的宏,它对其参数的约束也就越多,并且你利用
这些约束来产生高效的代码的机会也就越好。

%%% Local Variables:
%%% coding: utf-8
%%% mode: latex
%%% TeX-master: "onlisp-cn"
%%% End:
