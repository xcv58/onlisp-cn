%!TEX encoding = UTF-8 Unicode
% $Id: 9-variable_capture.tex 18 2014-03-12 22:35:24Z binghe $

\chapter{变量捕捉}
\label{chap:variable_capture}
\index{capture 捕捉}

宏很容易遇到一类被称为\emph{变量捕捉}\index{variable capture 变量捕捉|see {capture}}%
的问题。变量捕捉发生在宏展开导致名字冲突的时候,名字冲突指:某些符号结果出乎意料
地引用了来自另一个上下文中的变量。无意的变量捕捉可能会造成极难发觉的
~bug\index{macros 宏!errors in|see {capture}}。
本章将介绍预见和避免它们的办法。不过,有意的变量捕捉却也是一种有用的
编程技术,而且第~\ref{chap:anaphoric_macros} 章的宏都是靠这种技术实现的。

\section{宏参数捕捉}
\label{sec:macro_argument_capture}
\index{capture 捕捉!macro argument capture 宏参数捕捉}

如果一个宏对无意识的变量捕捉毫无防备,那么它就是有~bug 的宏。为了避免写出这样的宏,
我们必须确切地知道捕捉发生的时机。变量捕捉可以分为两类情况:
宏参数捕捉和自由符号捕捉。所谓参数捕捉,就是在宏调用中作为参数传递的符号无意地
引用到了宏展开式本身建立的变量。考虑下面这个~\texttt{for} 宏的定义,它像
~Pascal 的 \texttt{for} 在一系列表达式上循环操作:
\begin{lstlisting}
(defmacro for ((var start stop) &body body)
  `(do ((,var ,start (1+ ,var))
        (limit ,stop))
       ((> ,var limit))
     ,@body))
\end{lstlisting}
这个宏乍看之下没有问题。它甚至似乎也可以正常工作:
\begin{lstlisting}
> (for (x 1 5)
    (princ x))
12345
NIL
\end{lstlisting}
确实,这个错误如此隐蔽,可能用上这个版本的宏数百次,都毫无问题。
但如果我们这样调用它,问题就出来了:
\begin{lstlisting}
(for (limit 1 5)
  (princ limit))
\end{lstlisting}
我们可能会认为这个表达式和之前的结果相同。但它却没有任何输出:它产生了
一个错误。为了找到原因,我们仔细观察它的展开式:
\begin{lstlisting}
(do ((limit 1 (1+ limit))
     (limit 5))
    ((> limit limit))
  (print limit))
\end{lstlisting}
现在错误的地方就很明显了。在宏展开式本身的符号和作为参数传递给宏的符号之间出现
了名字冲突。宏展开\emph{捕捉}了~\texttt{limit}。这导致它在同一个~\texttt{do}
里出现了两次,而这是非法的。

由变量捕捉导致的错误比较罕见,但频率越低其性质就越恶劣。上个捕捉相对还比较温
和\pozhehao{}至少这次我们得到了一个错误。更普遍的情况是,捕捉了变量的宏
只是产生错误的结果,却没有给出任何迹象显示问题的源头。在下面的例子中,
\begin{lstlisting}
> (let ((limit 5))
    (for (i 1 10)
      (when (> i limit)
        (princ i))))
NIL
\end{lstlisting}
产生的代码静悄悄地什么也不做。\label{example:capture}

\section{自由符号捕捉}
\label{sec:free_symbol_capture}
\index{capture 捕捉!free symbol capture 自由符号捕捉}

偶尔会出现这样的情况,宏定义本身有这么一些符号,它们在宏展开时无意中却引用到了
其所在环境中的绑定。假设有个程序,它希望把运行中产生的警告信息保存在一个列表里
供事后检查,而不是在问题发生时直接打印输出给用户。于是有人写了一个宏
~\texttt{gripe}\label{macro:gripe},它接受一个警告信息,并把它加入全局列表
~\texttt{w}:
\begin{lstlisting}[escapechar=\~]
(defvar w nil)

(defmacro gripe (warning)       ~\hfill~; wrong
  `(progn (setq w (nconc w (list ,warning)))
          nil))
\end{lstlisting}
之后,另一个人希望写个函数~\texttt{sample-ratio},用来返回两个列表的长度比。如
果任何一个列表中的元素少于两个,函数就改为返回~\texttt{nil},同时产生一个警告说
明这个函数处理的是一个统计学上没有意义的样本。(实际的警告本可以带有更多的信
息,但它们的内容与本例无关。)
\begin{lstlisting}
(defun sample-ratio (v w)
  (let ((vn (length v)) (wn (length w)))
    (if (or (< vn 2) (< wn 2))
        (gripe "sample < 2")
        (/ vn wn))))
\end{lstlisting}
如果用~\texttt{w = (b)} 来调用~\texttt{sample-ratio},那么它将会警告说它有个
参数只含一个元素,因而得出的结果从统计上来讲是无意义的。但是当对~\texttt{gripe} 的调用被展开
时,\texttt{sample-ratio} 就好像被定义成:
\begin{lstlisting}
(defun sample-ratio (v w)
  (let ((vn (length v)) (wn (length w)))
    (if (or (< vn 2) (< wn 2))
        (progn (setq w (nconc w (list "sample < 2")))
               nil)
        (/ vn wn))))
\end{lstlisting}
这里的问题是,使用~\texttt{gripe} 时的上下文含有~\texttt{w} 自己的局部绑定。所以,
产生的警告没能保存到全局的警告列表里,而是被~\texttt{nconc} 连接到了
~\texttt{sample-ratio} 的一个参数的结尾。不但警告丢失了,而且列表~\texttt{(b)}
也加上了一个多余的字符串,而程序的其他地方可能还会把它作为数据继续使用:
\begin{lstlisting}
> (let ((lst '(b)))
    (sample-ratio nil lst)
    lst)
(B "sample < 2")
> w
NIL
\end{lstlisting}

\section{捕捉发生的时机}
\label{sec:when_capture_occurs}

许多宏的编写者都希望通过查看宏的定义,就可以预见到所有可能来自上述两种捕捉类型
的问题。变量捕捉有些难以捉摸,需要一些经验才能预料到那些被捕捉的变量在程序中
所有捣乱的伎俩。幸运的是,还是有办法在你的宏定义中找出那些可能被捕捉的符号\index{capture 捕捉!detecting potential},并排除它们的,
而无需操心这些符号捕捉\emph{如何}搞砸你的程序。本节将介绍一套直接了当的检测原则,
用它就可以找出可捕捉的符号。本章的其余部分则解释了避免出现变量捕捉的相关技术。

我们接下来提出的方法可以用来定义可捕捉的变量,但是它基于几个从属的概念,所以
在继续之前必须首先给这些概念下个定义:

\begin{defineit}{自由~(free)}\index{variables 变量!free 自由}
我们认为表达式中的符号~\emph{s} 是自由的,当且仅当它
被用作表达式中的变量,但表达式却没有为它创建一个绑定。
\end{defineit}

在下列表达式里,
\begin{lstlisting}
(let ((x y) (z 10))
  (list w x z))
\end{lstlisting}
\texttt{w}, \texttt{x} 和~\texttt{z} 在~\texttt{list} 表达式中看上去都是自由的,
因为这个表达式没有建立任何绑定。不过,外围的~\texttt{let} 表达式为~\texttt{x} 和
~\texttt{z} 创建了绑定,从整体上说,在~\texttt{let} 里面,只有~\texttt{y} 和
~\texttt{w} 是自由的。注意到在
\begin{lstlisting}
(let ((x x))
  x)
\end{lstlisting}
里~\texttt{x} 的第二个实例是自由的\pozhehao{}因为它并不在为~\texttt{x} 创建的新绑定
的作用域内。

\begin{defineit}{框架~(skeleton)}\index{macros 宏!skeletons of}
宏展开式的框架是整个展开式,并且去掉任何在宏调用中作为实参的部分。
\end{defineit}

如果~\texttt{foo} 的定义是:
\begin{lstlisting}
(defmacro foo (x y)
  `(/ (+ ,x 1) ,y))
\end{lstlisting}
并且被这样调用:
\begin{lstlisting}
(foo (- 5 2) 6)
\end{lstlisting}
那么它就会产生如下的展开式:
\begin{lstlisting}
(/ (+ (- 5 2) 1) 6)
\end{lstlisting}
这一展开式的框架就是上面这个表达式在把形参~\texttt{x} 和~\texttt{y} 拿走,留下空白
后的样子:
\begin{lstlisting}
(/ (+         1)  )
\end{lstlisting}

有了这两个概念,就可以把判断可捕捉符号的方法简单表述如下:

\begin{defineit}{可捕捉~(capturable)}
如果一个符号满足下面条件之一,那就可以认为它在某些宏展开里是可捕捉的:(a)
它作为自由符号出现在宏展开式的框架里,或者~(b) 它被绑定到框架的一部分,而该框架中
含有传递给宏的参数,这些参数被绑定或被求值。
\end{defineit}

用些例子可以明确这个标准的含义。在最简单的情况下:
\begin{lstlisting}
(defmacro cap1 ()
  '(+ x 1))
\end{lstlisting}
\texttt{x} 可被捕捉是因为它作为自由符号出现在框架里。 这就是导致~\texttt{gripe}
中~bug 的原因。在这个宏里:
\begin{lstlisting}
(defmacro cap2 (var)
  `(let ((x ...)
         (,var ...))
     ...))
\end{lstlisting}
\texttt{x} 可被捕捉是因为它被绑定在一个表达式里,而同时也有一个宏调用的参数被绑定了。
(这就是~\texttt{for} 中出现的错误。) 同样对于下面两个宏:
\begin{lstlisting}
(defmacro cap3 (var)
  `(let ((x ...))
     (let ((,var ...))
       ...)))

(defmacro cap4 (var)
  `(let ((,var ...))
     (let ((x ...))
       ...)))
\end{lstlisting}
\texttt{x} 在两个宏里都是可捕捉的。然而,如果~\texttt{x} 的绑定和作为参数传递的变量
没有这样一个上下文,在这个上下文中,两者是同时可见的,就像在这个宏里:
\begin{lstlisting}
(defmacro safe1 (var)
  `(progn (let ((x 1))
            (print x))
          (let ((,var 1))
            (print ,var))))
\end{lstlisting}
那么~\texttt{x} 将不会被捕捉到。并非所有绑定在框架里的变量都是有风险的。尽管如此,如果宏
调用的参数在一个由框架建立的绑定里被求值,
\begin{lstlisting}
(defmacro cap5 (&body body)
  `(let ((x ...))
     ,@body))
\end{lstlisting}
那么,这样绑定的变量就有被捕捉的风险:在~\verb|cap5| 中,\verb|x| 是可捕捉的。不过
对于下面这种情况,
\begin{lstlisting}
(defmacro safe2 (expr)
  `(let ((x ,expr))
     (cons x 1)))
\end{lstlisting}
\texttt{x} 是不可捕捉的,因为当传给~\verb|expr| 的参数被求值时,\verb|x| 的新绑定
将是不可见的。同时,请注意我们只需关心那些框架变量的绑定。在这个宏里
\begin{lstlisting}
(defmacro safe3 (var &body body)
  `(let ((,var ...))
     ,@body))
\end{lstlisting}
没有符号会因没有防备而被捕捉~(假设第一个参数的绑定是用户有意为之)。

现在让我们来检查一下~\texttt{for} 最初的定义,看看使用新的规则是否能发现可捕捉的符号:
\begin{lstlisting}[escapechar=\~]
(defmacro for ((var start stop) &body body)             ~\hfill~; wrong
  `(do ((,var ,start (1+ ,var))
        (limit ,stop))
       ((> ,var limit))
     ,@body))
\end{lstlisting}
现在可以看出~\texttt{for} 的这一定义可能遭受\emph{两种}方式的捕捉:\texttt{limit}
可能会被作为第一个参数传给~\texttt{for},就像在最早的例子里那样:
\begin{lstlisting}
(for (limit 1 5)
  (princ limit))
\end{lstlisting}
但是,如果~\texttt{limit} 出现在循环体里,也同样危险:
\begin{lstlisting}
(let ((limit 0))
  (for (x 1 10)
    (incf limit x))
  limit)
\end{lstlisting}
这样用~\texttt{for} 的人,可能会期望他自己的~\texttt{limit} 绑定就是在循环里
递增的那个,最后整个表达式返回~55;事实上,只有那个由展开式框架生成的~\texttt{limit}
绑定会递增:
\begin{lstlisting}
(do ((x 1 (1+ x))
     (limit 10))
    ((> x limit))
  (incf limit x))
\end{lstlisting}
并且,由于迭代过程是由这个变量控制的,所以循环甚至将无法终止。

本节中介绍的这些规则不过是个参考,在实际编程中仅仅具有指导意义。它们甚至不是形式化定义的,
更不能完全保证其正确性。捕捉是一个不能明确定义的问题,它依赖于你期望的行为。例如,在下面的表达式里,
\begin{lstlisting}
(let ((x 1)) (list x))
\end{lstlisting}
\texttt{x} 在 \texttt{(list x)} 被求值时,会指向新的变量,不过我们不会把它视为错误。这正是
~\texttt{let} 要做的事。检测捕捉的规则也含混不清。你可以写出通过这些测试的宏,而这样的宏
却仍然有可能会遭受意料之外的捕捉。例如,
\begin{lstlisting}[escapechar=\~]
(defmacro pathological (&body body)                     ~\hfill~; wrong
  (let* ((syms (remove-if (complement #'symbolp)
                          (flatten body)))
         (var (nth (random (length syms))
                   syms)))
    `(let ((,var 99))
       ,@body)))
\end{lstlisting}
当调用这个宏的时候,宏主体中的表达式就像是在一个~\texttt{progn} 中被求值\pozhehao{}
但是主体中有一个随机选出的变量将带有一个不同的值。这很明显是一个捕捉,但它通过了我们的测试,
因为这个变量并没有出现在框架里。然而,实践表明该规则在绝大多数时候都是正确的:很少有人
~(如果真有的话) 会想写出类似上面那个例子的宏。

\section{取更好的名字避免捕捉}
\label{sec:avoiding_capture_with_better_names}

前两节将变量捕捉分为两类:参数捕捉,在这种情况下,由宏框架建立的绑定会捕捉参数中用到的符号;
和自由符号捕捉,而在这里,宏展开处的绑定会捕捉到宏展开式中的自由符号。
常常可以通过给全局变量取个明显的名字来解决后一类问题\index{capture 捕捉!free symbol capture 自由符号捕捉!avoiding 避免}。在~Common Lisp 中,习惯上会给全局变量\index{variables 变量!global 全局}取一个两头都是星号的名字。例如,定义当前包的变量叫做~\texttt{package}\index{package@\texttt{*package*}}。(这样的名字可以发音为~``star--package--star'' 来强调它不是普通的变量。)

所以~\texttt{gripe} 的作者的的确确有责任把那些警告保存在一个名字类似~\texttt{*warnings*} 而非~\texttt{w} 的变量中。
如果~\texttt{sample-ratio} 的作者执意要用~\texttt{*warnings*} 做函数参数,那他碰到的每个~bug 都是咎由自取,但如果他觉得用
~\texttt{w} 作为参数的名字应该比较保险,就不应该再怪他了。

\section{通过预先求值避免捕捉}
\label{sec:avoiding_capture_by_prior_evaluation}
\index{capture 捕捉!avoiding by prior evaluation}

\begin{figure}
易于被捕捉的:
\begin{lstlisting}
(defmacro before (x y seq)
  `(let ((seq ,seq))
     (< (position ,x seq)
        (position ,y seq))))
\end{lstlisting}
一个正确的版本:
\begin{lstlisting}
(defmacro before (x y seq)
  `(let ((xval ,x) (yval ,y) (seq ,seq))
     (< (position xval seq)
        (position yval seq))))
\end{lstlisting}
\caption{\label{fig:avoiding_capture_with_let}用~\texttt{let} 避免捕捉}
\end{figure}

有时,如果不在任何宏展开创建的绑定里求值那些有危险的参数,就可以轻松消除参数捕捉。
最简单的情况可以这样处理:让宏以~\texttt{let} 表达式开头。图
~\ref{fig:avoiding_capture_with_let} 包含宏~\texttt{before} 的两个版本,该宏接受
两个对象和一个序列,当且仅当第一个对象在序列中出现于第二个对象之前时返回真。\footnote{%
这个宏只是个例子。实际编程中,它既不应当实现成宏,也不该用这种低效的算法。若需要正确的定义,可见
第~\pageref{fig:functions_which_search_lists} 页。}
第一个定义是不正确的。它开始的~\texttt{let} 确保了作为~\texttt{seq} 传递的~form
只求值一次,但是它不能有效地避免下面这个问题:
\begin{lstlisting}
> (before (progn (setq seq '(b a)) 'a)
          'b
          '(a b))
NIL
\end{lstlisting}
这相当于问“\texttt{(a b)} 中的~\verb|a| 是否在~\verb|b| 前面?”如果
~\verb|before| 是正确的,它将返回真。宏展开式揭示了真相:对~\verb|<| 的
第一个参数的求值重新排列了那个将在第二个参数里被搜索的列表。
\begin{lstlisting}
(let ((seq '(a b)))
  (< (position (progn (setq seq '(b a)) 'a)
               seq)
     (position 'b seq)))
\end{lstlisting}
要想避免这个问题,只要在一个巨大的~\texttt{let} 里求值所有参数就行了。这样图
~\ref{fig:avoiding_capture_with_let} 中的第二个定义对于捕捉就是安全的了。

不幸的是,这种~\texttt{let} 技术只能在很有限的一类情况下才可行:
\begin{enumerate}
\item 所有可能被捕捉的参数都只求值一次,并且
\item 没有一个参数需要在宏框架建立的绑定下被求值。
\end{enumerate}
这个规则排除了相当多的宏。我们比较赞成的~\texttt{for} 宏就同时违反了这两个限制。然而,
我们可以把这个技术加以变化,使类似~\texttt{for} 的宏免于发生捕捉,即将其~\texttt{body} forms 
包装在一个\lexpr{}里,同时让这个\lexpr{}位于任何局部创建的绑定之外。

有些宏~(其中包括用于迭代的宏),如果宏调用里面有表达式出现,那么在宏展开
后,这些表达式将会在一个新建的绑定中求值。例如在~\texttt{for} 的定义
中,循环体必须在一个由宏创建的~\texttt{do} 中进行求值。因
此,\texttt{do} 创建的变量绑定会很容易就捕捉到循环里的变量。我们可以把
循环体包在一个闭包里,同时在循环里,不再把直接插入表达式,而只是简单
地~funcall 这个闭包。通过这种办法来保护循环中的变量不被捕捉。

\begin{figure}
易于被捕捉的:
\begin{lstlisting}
(defmacro for ((var start stop) &body body)
  `(do ((,var ,start (1+ ,var))
        (limit ,stop))
       ((> ,var limit))
     ,@body))
\end{lstlisting}
正确的版本:
\begin{lstlisting}
(defmacro for ((var start stop) &body body)
  `(do ((b #'(lambda (,var) ,@body))
        (count ,start (1+ count))
        (limit ,stop))
       ((> count limit))
     (funcall b count)))
\end{lstlisting}
  \caption{用闭包避免捕捉}
  \label{fig:avoiding_capture_with_closure}
\end{figure}

图~\ref{fig:avoiding_capture_with_closure} 给出了一个~\texttt{for} 的实现,它使用的就是这种技术\note{126}。
由于闭包是~\texttt{for} 展开时生成的第一个东西,因此,所有出现在宏体内的自由符号将全部指向
宏调用环境中的变量。现在~\texttt{do} 通过闭包的参数跟宏体通信。闭包需要从~\texttt{do}
知道的全部就是当前迭代的数字,所以它只有一个参数,也就是宏调用中作为索引指定的那个符号。

这种将表达式包装进~lambda 的方法也不是万金油。虽然你可以用它来保护代码体,但闭包有时也
起不到任何作用,例如,当存在同一变量在同一个~\texttt{let} 或~\texttt{do} 里被绑定两次
的风险时~(就像开始的那个有缺陷的~\texttt{for} 那样)。幸运的是,在这种情况下,通过重写
~\texttt{for} 将其主体包装在一个闭包里,我们同时也消除了~\texttt{do} 为~\texttt{var}
参数建立绑定的需要。原先那个~\texttt{for} 中的~\texttt{var} 参数变成了闭包的参数并且在
~\texttt{do} 里面可以被一个实际的符号~\texttt{count} 替换掉。所以这个~\texttt{for}
的新定义对于捕捉是完全免疫的,就像~\ref{sec:when_capture_occurs} 节里的测试所显示的那
样。

闭包的缺点在于,它们的效率可能不大理想。我们可能会因此造成又一次函数调用。更糟糕的是,如果
编译器没有给闭包分配动态作用域~(dynamic extent)\index{dynamic extent 动态作用域}\index{extent, dynamic},
那么一等到运行期,闭包所需的空间将不得不从堆里分配。
\footnote{译者注:dynamic extent 是一种~Lisp 编译器优化技术,详情请见
~\href{http://www.lisp.org/HyperSpec/Body/dec_dynamic-extent.html\#dynamic-extent}{Common Lisp HyperSpec} 的有关内容。}

\section{通过~gensym 避免捕捉}
\label{sec:avoiding_capture_with_gensyms}
\index{capture 捕捉!avoiding with gensyms}

这里有一种切实可行的方法可供避免宏参数捕捉:把可捕捉的符号换成~gensym。在
~\texttt{for} 的最初版本中,当两个符号意外地重名时,就会出问题。如果我们想要避免这种情况:宏框架里
含有的符号也同时出现在了调用方代码里,我们也许会给宏定义里的符号取个怪异的名字,寄希望以此来摆
脱参数捕捉的魔爪:
\begin{lstlisting}[escapechar=\~]
(defmacro for ((var start stop) &body body)             ~\hfill~; wrong
  `(do ((,var ,start (1+ ,var))
        (xsf2jsh ,stop))
       ((> ,var xsf2jsh))
     ,@body))
\end{lstlisting}
但是这治标不治本。它并没有消除~bug,只是降低了出问题的可能性。并且还有一个可能性不那么小的问题悬而未决
\pozhehao{}不难想象,如果把同一个宏嵌套使用的话,仍会出现名字冲突。

我们需要一个办法来\emph{确保}符号都是唯一的。Common Lisp 函数~\texttt{gensym}\index{gensym@\texttt{gensym}}
的意义正是在于此。\note{128}它返回的符号称为~\emph{gensym},这个符号可以保证不和任何手工
输入或者由程序生成的符号相等(\texttt{eq})。

那 Lisp 是如何保证这一点的呢?在~Common Lisp 中,每个包都维护着一个列表,用于保存这个包知道的所有符号。
(关于包~(package) 的介绍,可见~\pageref{chap:packages} 页。) 一个符号,只要出现在这个列表上,
我们就说它被\emph{\intern~(intern)}\index{intern@\texttt{intern}} 在这个包里。每次调用~\texttt{gensym} 都会
返回唯一,\unintern{}的符号。而~\texttt{read}\index{read@\texttt{read}} 每见到一个符号,都会把它\intern{}\index{interning},
所以没人能输入和~gensym 相同的东西。也就是说,如果你有个表达式是这样开头的
\begin{lstlisting}
(eq (gensym) ...
\end{lstlisting}
那么将无法让这个表达式返回真。

让~\texttt{gensym} 为你构造符号,这个办法其实和``选个怪名字''的方法异曲同工,而且更进一步
\pozhehao{}\texttt{gensym} 给你的名字甚至在电话薄里也找不到。如果~Lisp 不得不显示~gensym,
\begin{lstlisting}
> (gensym)
#:G47
\end{lstlisting}
它打印出来的东西基本上就相当于~Lisp 的“张三”,即为那种名字无关紧要的东西
编造出来的毫无意义的名字。并且为了确保我们不会对此有任何误会,gensym 在显示时候,前面加了一个井号--%
冒号\index{\#:@\texttt{\#:}},这是一种特殊的读取宏~(read--macro),其目的是为了让我们在试图第二次读取该~gensym 时报错。

在~\textsc{cltl}2 Common Lisp 里,gensym 的打印形式\index{print-names}\index{symbols 符号!names of}中的数字来自
~\texttt{*gensym-counter*}\index{gensym-counter@\texttt{*gensym-counter*}}\index{Common Lisp!differences between versions!*gensym-counter*@\texttt{*gensym-counter*}},这个全局变量总是绑定到某个整数。如果重置这个计数器,我们就可以让两个~gensym 的打印输出一模一样
\begin{lstlisting}
> (setq x (gensym))
#:G48
> (setq *gensym-counter* 48 y (gensym))
#:G48
> (eq x y)
NIL
\end{lstlisting}
但它们不是一回事。

\begin{figure}
易于被捕捉的:
\begin{lstlisting}
(defmacro for ((var start stop) &body body)
  `(do ((,var ,start (1+ ,var))
        (limit ,stop))
       ((> ,var limit))
     ,@body))
\end{lstlisting}
一个正确的版本:
\begin{lstlisting}
(defmacro for ((var start stop) &body body)
  (let ((gstop (gensym)))
    `(do ((,var ,start (1+ ,var))
          (,gstop ,stop))
         ((> ,var ,gstop))
       ,@body)))
\end{lstlisting}
\caption{\label{fig:avoiding_capture_with_gensym}用~gensym 避免捕捉}
\end{figure}

图~\ref{fig:avoiding_capture_with_gensym} 中有一个使用~gensym 的~\verb|for|
的正确定义。现在就没有~\verb|limit| 可以和传进宏的~form 里的符号有冲突了。它已经被
换成一个在现场生成的符号。宏每次展开的时候,\verb|limit| 都会被一个在展开期
创建的唯一符号取代。

初次就把~\verb|for|\label{macro:for} 定义得完美无缺,还是很难的。
完成后的代码,如同一个完成了的定理,精巧漂亮的证明的背后是一次次的尝试和失败。
所以不要担心你可能会对一个宏写好几个版本。在开始写类似
~\texttt{for} 这样的宏时,你可以在不考虑变量捕捉问题的情况下,先把第一个版本写出来,然后再回过头来
为那些可能卷入捕捉的符号制作~gensym。

\section{通过包避免捕捉}
\label{sec:avoiding_capture_with_packages}
\index{capture 捕捉!avoiding with packages}
\index{packages!avoiding capture with}

从某种程度上说,如果把宏定义在它们自己的包里,就有可能避免捕捉。倘若你创建一个
~\texttt{macros} 包,并且在其中定义~\texttt{for},那么你甚至可以使用最初给出的定义
\begin{lstlisting}
(defmacro for ((var start stop) &body body)
  `(do ((,var ,start (1+ ,var))
        (limit ,stop))
       ((> ,var limit))
     ,@body))
\end{lstlisting}
这样,就可以毫无顾虑地从其他任何包调用它。如果你从另一个包,比方说~\texttt{mycode},里调用
~\texttt{for},就算把~\texttt{limit} 作为第一个参数,它也是
\texttt{mycode::limit}\pozhehao{}
这和~\texttt{macros::limit} 是两回事,后者才是出现在宏框架中的符号。

然而,包还是没能为捕捉问题提供面面俱到的通用解决方案。首先,宏是某些程序不可或缺
的组成部分,将它们从自己的包里分离出来会很不方便。其次,这种方法无法为~\texttt{macros}
包里的其他代码提供任何捕捉保护。

\section{其他名字空间里的捕捉}
\label{sec:capture_in_other_name-spaces}

前面几节都把捕捉说成是一种仅影响变量的问题。尽管多数捕捉都是变量捕捉,但是
~\texttt{Common Lisp} 的其他名字空间里也同样会有这种问题。

函数也可能在局部被绑定,因而,函数绑定也会因无意的捕捉而导致问题。例如,
\begin{lstlisting}
> (defun fn (x) (+ x 1))
FN
> (defmacro mac (x) `(fn ,x))
MAC
> (mac 10)
11
> (labels ((fn (y) (- y 1)))
    (mac 10))
9
\end{lstlisting}
正如捕捉规则预料的那样,以自由之身出现在~\verb|mac| 框架中的~\verb|fn| 带来了被捕捉的风险。如果
~\verb|fn| 在局部被重新绑定的话,那么~\verb|mac| 的返回值将和平时不一样。

对于这种情况,该如何应对呢?当有捕捉风险的符号与内置函数或宏重名时,那么听之任之应该是上策。
\note{131}\textsc{cltl}2 (260 页) 说,如果任何内置的名字被用作局部函数或宏绑定,``后果
是未定义的。''\index{Common Lisp!differences between versions!redefining built-in operators} 所以你的宏无论做了什么都没关系\pozhehao{}任何人,如果重新绑定内置函数,那么他将来碰到的问题会比
你的这个宏更多\index{functions 函数!redefining built-in}。

另一方面,保护变量名的方法同样可以用来帮助函数名\index{capture 捕捉!of function names}免于宏参数捕捉:通过使用~gensym 作为
宏框架局部定义的任何函数的名字。但是,如果要避免像上面这种情况中的自由符号捕捉,就会稍微麻烦一点。要让变量免受
自由符号捕捉,采用的保护方法是使用一目了然的全局名称:例如把~\verb|w| 换成~\verb|*warnings*|
。然而,这个解决方案对函数有些不切实际,因为没有把全局函数的名字区分出来的习惯\pozhehao{}大多数函数都是全局
的。如果你担心发生这种情况,一个宏使用了另一个函数,而调用这个宏的环境可能会重定义这个函数,那么最佳的解决方案或许就是
把你的代码放在一个单独的包里。\index{packages!avoiding capture with}\index{packages!using distinct}

代码块名字~(block-name)\index{block-names 代码块名字} 同样可以被捕捉\index{capture 捕捉!of block names},比如说那些被~\texttt{go} 和~\texttt{throw}
使用的标签~(tag)\index{capture 捕捉!of tags}。当你的宏需要这些符号时,你应该像
~\pageref{fig:implementing_do} 页上~\texttt{our-do} 的定义那样,使用~gensym。

还需要注意的是像~\texttt{do} 这样的操作符隐式封装在一个名为~\texttt{nil}\index{nil@\texttt{nil}!default block name} 的块\index{do@\texttt{do}!implicit block@\texttt{block} in}里。这样在
~\texttt{do} 里面的一个~\texttt{return}\index{return@\texttt{return}} 或~\texttt{return-from nil}\index{return-from@\texttt{return-from}} 将从
~\texttt{do} 本身而非包含这个~\texttt{do} 的表达式里返回:\label{block_and_capture}
\begin{lstlisting}
> (block nil
    (list 'a
          (do ((x 1 (1+ x)))
              (nil)
            (if (> x 5)
                (return-from nil x)
                (princ x)))))
12345
(A 6)
\end{lstlisting}
如果~\texttt{do} 没有创建一个名为~\texttt{nil} 的块,这个例子将只返回~\texttt{6},而
不是~\texttt{(A 6)}。

\texttt{do} 里面的隐式块\index{block@\texttt{block}!implicit 隐式}不是问题,因为~\texttt{do} 的这种工作方式广为人知。尽管如此,如果
你写一个展开到~\texttt{do} 的宏,它将捕捉~\texttt{nil} 这个块名称。在一个类似
~\texttt{for} 的宏里,\texttt{return} 或~\texttt{return-from nil} 将从
~\texttt{for} 表达式而非封装这个~\texttt{for} 表达式的块中返回。

\section{为何要庸人自扰?}
\label{sec:why_bother}

前面举的例子中有些非常牵强做作。看着它们,有人可能会说,“变量捕捉既然这么少见\pozhehao{}
为什么还要操心它呢?”回答这个问题有两个方法。一个是用另一个问题反诘道:要是你写得出没有
~bug 的程序,为什么还要写有小~bug 的程序呢?

更长的答案是指出在现实应用程序中,对你代码的使用方式做任何假设都是危险的。任何~Lisp 程序都具备现在被称之为“开放式架构”的特征。如果你正在写的代码以后会为他人所用,很可能他们调用你代码的方式是出乎你预料的。而且你要担心的不光是人。程序也能编写程序。可能没人会写这样的代码
\begin{lstlisting}
(before (progn (setq seq '(b a)) 'a)
        'b
        '(a b))
\end{lstlisting}
但是程序生成的代码看起来经常就像这样。即使单个的宏生成的是简单合理的展开式,
一旦你开始把宏嵌套着调用,展开式就可能变成巨大的,而且看上去没人能写得出来的程序。在这个前提下,就有必要去预防那些可能使你的宏不正确地展开的情况,就算这种情况像是有意设计出来的。

最后,避免变量捕捉不管怎么说,并非难于上青天。它很快会成为你的第二直觉。Common Lisp 
中经典的~\verb|defmacro| 好比厨子手中的菜刀:美妙的想法看上去会有些危险,
但是这件利器一到了专家那里,就如入庖丁之手,游刃有余。

%%% Local Variables:
%%% coding: utf-8
%%% mode: latex
%%% TeX-master: "onlisp-cn"
%%% End:
